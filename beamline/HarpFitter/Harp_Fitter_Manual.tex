\documentclass[letterpaper,12pt]{article}
%documentclass[superscriptaddress,preprintnumbers,amsmath,amssymb,aps,11pt]{revtex4}
%\usepackage[]{authblk}
%\usepackage{graphics}
\usepackage[dvipdf]{graphicx}
%\usepackage{subfig}  % For subfloats
\usepackage{color}
\usepackage[usenames,dvipsnames]{xcolor}
\usepackage{epsfig}
\usepackage{wrapfig}
\usepackage{rotating}
\usepackage{caption}
%\usepackage{subcaption}
\usepackage{subfig}
\usepackage{authblk}
%\usepackage{url}
\usepackage{hyperref}
\usepackage{nameref}
%\usepackage{eqnarray,amsmath}7

\renewcommand{\thesubsection}{\thesection.\alph{subsection}}

\oddsidemargin = -14mm
\topmargin = -2.9cm
\textwidth = 19cm
\textheight = 24cm

\def \rarr {\rightarrow}
\def \grinp {\includegraphics}
\def \tw {\textwidth}
\def \dfrac#1#2{\displaystyle{{#1}\over{#2}}}
\def \dstl {\displaystyle}
\def \Apr {A^{\prime}}

\title{Harp Fitter manual}
\author{Rafayel Paremuzyan}

\begin{document}
\maketitle

For performing the harp scan, follow instructions in chapter 3.5 of the beamline manual.
The analyzer can be open from the harp scan GUI using ``Analyze" button.
The general view of the fitter application is shown in Fig. \ref{fig:Gen_harp_gen_View}. On the main screen there are 
7 buttons, names of each button gives hint to it's function:
%%%%%%%%%%%%%%%%%%%%%%%%%%%%%%%%%%%%%%%%% F I G U R E %%%%%%%%%%%%%%%%%%%%%%%%%%%%%%%%%%%%%%%%%%%%%
 \begin{figure}[!htb]
  \centering
  \subfloat[]{\label{fig:Gen_harp}\grinp[width=0.45\tw]{img/Harp_SCan_CSS.png}} \hskip 1cm
  \subfloat[]{\label{fig:Gen_View}\grinp[width=0.45\tw]{img/General_View.png}}
  %\grinp[width=0.45\tw]{img/General_View.png}
  \caption{Left: The harp scan GUI. Right: General view of the fitter application.}
  \label{fig:Gen_harp_gen_View}
 \end{figure}
%%%%%%%%%%%%%%%%%%%%%%%%%%%%%%%%%%%%%%%%% F I G U R E %%%%%%%%%%%%%%%%%%%%%%%%%%%%%%%%%%%%%%%%%%%%%
\begin{itemize}
 \item "Choose a File" - Allows selection of harp scan file, if file is not selected, analyzer will fit the
 latest scan.
 \item "View data from All counters" - pops up a window where user can see responces from all counters. This is useful if you want
 to check, if harp scan went correctly.
 \item "Choose a counter to Fit" - Allows the user to select a counter, for fitting.
 \item "Fit Data" - Performs the fitting of the selected counter.
 \item "Submit to Logbook" - Allows the user to send fit results to logbook, and updata MYA database with fit results.
 \item "Set Fit Ranges" - Allows the user to manually adjust fit ranges for each wire. This is helpfull, when the standart
 ``Fit Data" function fails to fit data.
 \item "Exit" - Exits the program.
\end{itemize}
\section{Fitting the data}
If you have the screen on Fig.\ref{fig:Gen_View} open, then to analyze recent harp scan data, 
you first need to select the halo counter from the drop-down menu ``Choose a counter to Fit".
If you are not sure which halo counter is better for that particular harp, look into Section 
\ref{sec:suggested_counters} \textbf{"\nameref{sec:suggested_counters}"}.
As soon you select the halo counter push the "Fit Data" button, and if everything is alright, it should fit.
 If harp is 2c21, you should see 2 peaks, as shown in Fig.\ref{fig:2wire_FitExample}. If you fit tagger harp, or harp 2H01,
 then you should see 3 peaks as shown in Fig.\ref{fig:3wire_FitExample}.
%%%%%%%%%%%%%%%%%%%%%%%%%%%%%%%%%%%%%%%%% F I G U R E %%%%%%%%%%%%%%%%%%%%%%%%%%%%%%%%%%%%%%%%%%%%%
 \begin{figure}[!htb]
  \centering
  \subfloat[]{\label{fig:2wire_FitExample}\grinp[width=0.45\tw]{img/2c21_FitExample.png}} \hskip 1cm
  \subfloat[]{\label{fig:3wire_FitExample}\grinp[width=0.45\tw]{img/3wire_fit_Example.png}}
  \caption{Examples of fits. Left 2 wire harp shows only 2 peaks. Right an example of 3-wire harp data fit, tagger harp or 2H01 harp.
  Shold be 3 peaks.}
 \end{figure}
%%%%%%%%%%%%%%%%%%%%%%%%%%%%%%%%%%%%%%%%% F I G U R E %%%%%%%%%%%%%%%%%%%%%%%%%%%%%%%%%%%%%%%%%%%%%
 All harp scans (unless it is not usable) should be submitted to Logbook.
 
 \section{Submit the fit results to Logbook}
 To submit the analyzed results to logbook, push the "Submit to Logbook" button in the main window.
 %%%%%%%%%%%%%%%%%%%%%%%%%%%%%%%%%%%%%%%%% F I G U R E %%%%%%%%%%%%%%%%%%%%%%%%%%%%%%%%%%%%%%%%%%%%%
 \begin{figure}[!htb]
  \centering
  \grinp[width=0.35\tw]{img/MakeComments.png}
  \caption{Submit to logbook window, where you can make comments, chose which logbook to send, and also you can choose whether
  send fit paramaters into MYA database.}
  \label{fig:Submit_To_Logbook}
 \end{figure}
%%%%%%%%%%%%%%%%%%%%%%%%%%%%%%%%%%%%%%%%% F I G U R E %%%%%%%%%%%%%%%%%%%%%%%%%%%%%%%%%%%%%%%%%%%%%
A window similar to Fig.\ref{fig:Submit_To_Logbook} will pop-up. Make any relevant comments if needed,
then from check-boxes choose logbooks to send. By default "HBLog" is selected, which is the primary
logbook for Hall B experiments. Choosing ELog, the log entry will appear in the accelerator logbook as well.
"TLog" is logbook, which is used only for testing purposes. {\color{Blue} \bf ATTENTION: please check the "Send to MYA" box,
if you are analyzing the most recent harp scan data, otherwise don't check that button.}
 
 \section{Suggested counters} \label{sec:suggested_counters}
 These are suggested counters for each harp.
 \begin{itemize}
  \item for 2c21 harp, use Upstream-L or Upstream-R, Choose the one which shows less background/signal ratio.
  \item for harp tagger, when Tagger magnet is ``On", use Tagger-Left, Tagger-Raight or Tagger-Top, otherwise use
  Midstream-bot, Midstream-Left, Midstream-Right, Midstream-top. Choose the one which hass less background/signal ratio
  \item fot 2H01 harp, use Downstream-Left, Downstream-Right, Downstream-Bottop, or Downstream-Top, choose the one which hass less background/signal ratio.
 \end{itemize}
 
 \section{Problems with fit}
%  \begin{itemize}
%   \item If fit didn't work, i.e. you don't get 2 peaks (for 2c21) or 3 peaks (for tagger harp of 2H01 harp)
%  \end{itemize}
%  Please make sure you have selected the appropriate counter from the "Choose a counter to fit" drop down menu.
%  If that doesn't help:
 
%  \begin{itemize}
%   \item If that doesn't help:
%  \end{itemize}
%  push the "View Data from All counters". A new window will pop-up, where you will
%  see counts as a function of motor position for all counters. Please make sure that you have correct number of
%  peaks for corresponding harp. It is 2 peaks for 2c21 harp, and see 3 peaks for tagger and 2H02A harps,
%  if that is not the case, it means harp scan wasn't went correctly, might be for example the beam has tripped,
%  during the scan or another reason. You should do harp scan one more time, and make sure you have correct number of peaks.
 
 %%%%%%%%%%%%%%%%%%%%%%%%%%%%%%%%%%%%%%%%% F I G U R E %%%%%%%%%%%%%%%%%%%%%%%%%%%%%%%%%%%%%%%%%%%%%
\begin{figure}[!htb]
 \centering
 \grinp[width=0.45\tw]{img/ViewAllCounters.png}
 \caption{An example showing a window of Data from all counters.}
\end{figure}
 %%%%%%%%%%%%%%%%%%%%%%%%%%%%%%%%%%%%%%%%% F I G U R E %%%%%%%%%%%%%%%%%%%%%%%%%%%%%%%%%%%%%%%%%%%%%
 \begin{itemize}
  \item If it shows only one graph, and there is no fit in it,
 \end{itemize}
  This means in the data, the fitter didn't find correct number of peaks. First check
  with "View Data from All counters" button, if there are correct number of peaks (2 for 2c21, and 3 for tagger and 2H02A harps).
  If number of peaks is incorrect, this mean harp scan didn't go correctly, you should do scan again, and make sure you have correct number of peaks.
  {\color{Blue} \bf ATTENTION: if you don't exit from the application, then you have to load the new file using the
 "Choose a File" button. (see section \ref{sec:Choose_File}}\textbf{\nameref{sec:Choose_File}}, after then do a fit).
 However, If number of peaks is correct, but the fitter is not able to fit,
 try to set fit ranges manually. See section \ref{sec:Fit_Ranges} \textbf{\nameref{sec:Fit_Ranges}}
%  \begin{itemize}
%   \item However, If number of peaks is correct, but the fitter is not able to fit.
%  \end{itemize}
 
 \section{Loading a File to analyze} \label{sec:Choose_File}
 If you would like to analyze some particular file, push the button "Choose a file"
 %%%%%%%%%%%%%%%%%%%%%%%%%%%%%%%%%%%%%%%%% F I G U R E %%%%%%%%%%%%%%%%%%%%%%%%%%%%%%%%%%%%%%%%%%%%% 
 \begin{figure}[!htb]
  \centering
  \subfloat[]{\label{fig:choseFile1}\grinp[width=0.4\tw]{img/Chose_File1.png}} $\;\rarr$
  \subfloat[]{\label{fig:choseFile2}\grinp[width=0.45\tw]{img/Chose_File2.png}}
  \caption{Chose a harp scan data file to fit. Left: choose the directory of the harp. Right: chose your
  desired file and press "Open"}
  \label{fig:choseFile}
 \end{figure}
 %%%%%%%%%%%%%%%%%%%%%%%%%%%%%%%%%%%%%%%%% F I G U R E %%%%%%%%%%%%%%%%%%%%%%%%%%%%%%%%%%%%%%%%%%%%%  
 then as shown in the Fig.\ref{fig:choseFile}, 1st select the directory of the harp, that you want to analyze, then
 from that directory select the file, and open it. Now you can fit it with "Fit Data" button.

 \section{Setting Fit ranges Manually} \label{sec:Fit_Ranges}
 Sometimes it happens that the fitter is not able to find correct peaks. In this case 
 the user can give left and right limits of each peak, to aid the fit. In order to do that, you should push the
 button "Set Fit Ranges" in the main window. and a new window will pop-up which will look like the one in Fig.\ref{fig:SetFitRanges}.
 %%%%%%%%%%%%%%%%%%%%%%%%%%%%%%%%%%%%%%%%% F I G U R E %%%%%%%%%%%%%%%%%%%%%%%%%%%%%%%%%%%%%%%%%%%%% 
 \begin{figure}[!htb]
  \centering
  \grinp[width=0.65\tw]{img/SetFitRange.png}
  \caption{With this window user can set left and right limits for each peak.}
  \label{fig:SetFitRanges}
 \end{figure}
 %%%%%%%%%%%%%%%%%%%%%%%%%%%%%%%%%%%%%%%%% F I G U R E %%%%%%%%%%%%%%%%%%%%%%%%%%%%%%%%%%%%%%%%%%%%%
 Default values for left and right limits set such it will include the peak, however, the actual beam position may
 change, and the user should make sure that the peaks are inside the selected ranges, and then Fit with a "Fit"
 button, which is under ranges in that window. 
 \begin{itemize}
  \item If still it doesn't fit.
 \end{itemize}  
Try narrowing widening fit ranges, after few trials, if it doesn't work, then contact the slow-controls expert.
\end{document}