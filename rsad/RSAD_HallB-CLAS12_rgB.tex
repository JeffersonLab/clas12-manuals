\documentclass [12pt]{article}

\title{Radiological Safety Analysis Document for the Hall-B CLAS12 Run Group B run}

\begin{document}
\date{}
\vskip 0.5cm
\maketitle

\noindent

{\bf This Radiological Safety Analysis Document (RSAD) will identify the
general conditions associated with the CLAS12 
Run Group B (RG-B) run in Hall~B, and the controls associated with regard to production, 
movement, or import of radioactive materials.}\footnote{Contact person: Stepan Stepanyan}

\section{Description}
\indent

The physics run of the CLAS12 RG-B will take place in winter-spring of 2019 in experimental 
Hall~B. The CLAS12 is a multipurpose detector system based on a toroidal (forward detector) 
and a solenoid (central detector) superconducting magnets. The detector system includes Cherenkov Counters, 
Drift Chambers, Scintillator Counters, Silicon-strip detectors, Micro-mega gas detectors, 
and Calorimeters. For the RG-B, the CLAS12 detector will be complemented with a backward neutron detector (BAND). BAND is a set of scintillator counters that will be installed about $1.5$ meters upstream of the CLAS12 target. 

In this run period beams of up to 11~GeV 
(5 pass), and beam currents up to 100~nA will be used. As a target CLAS12 cryotarget  located inside 
the central detector in the center of the 5~T solenoid magnet will be used with and without LD$_2$ (liquid deuterium) and LH$_2$ (liquid hydrogen) in it. 
The target cell is a 20~mm diameter, 
5~cm long Kapton tube installed inside the beam vacuum, in a foam scattering chamber. The vacuum 
will be disconnected between the upstream and downstream beamlines with $\sim40$~cm of air between 
the exit window of the target scattering chamber and the entrance window of the downstream beamline. 
Both windows are 50~$\mu$m thick aluminum. The luminosity during the run will be 
$\sim 10^{35}$~cm$^{-2}$s$^{-1}$, the nominal design luminosity for CLAS12. During running with beam 
currents above 15~nA at 11~GeV ($\sim$160~W), the Hall~B beam stopper (a 30~cm long, water cooled-copper 
absorber) will be positioned before the Faraday Cup to prevent overheating. 

For the beam tune and M{\"o}ller runs, the beam will be directed into the beam dump in 
the Hall~B Tagger dipole yoke. On top of the Tagger yoke, additional lead and borated-poly shielding is added 
to limit radiation leakage to the CLAS12 detectors and electronics. The additional shielding 
and a nickel collimator have been thoroughly simulated using a GEANT model of the beamline and the 
Tagger dipole magnet. Downstream of the collimator box, additional shielding is installed from boratedpoly blocks to minimize neutron background.

% (see the expected neutron radiation dose on the electronics described in the Appendix).

\section{Summary and Conclusions}
\indent

The experiment is not expected to produce significant levels of radiation at the site boundary. 
However, it will be periodically monitored by the Radiation Control Department to ensure that the 
site boundary goal is not exceeded. The main consideration is the manipulation and/or handling of 
target(s) or beamline hardware. As specified in Sections IV (B) and VII, the manipulation and/or 
handling of target(s) or beamline hardware (potential radioactive material), the transfer of 
radioactive material, or modifications to the beamline after the target assembly must be reviewed 
and approved by the Radiation Control Department. Adherence to this RSAD is vital.


\section{Calculations of Radiation Deposited in the Experimental Hall
(the Experiment Operations Envelope)} 
\indent

The radiation budget for a given experiment is the amount of radiation that is expected at the 
site boundary as a result of a given set of experiments. This budget may be specified in terms 
of mrem at the site boundary or as a percentage of the Jefferson Lab design goal for dose to the 
public, which is 10~mrem per year. The Jefferson Lab design goal is 10\% of the DOE annual dose 
limit to the public, and cannot be exceeded without prior written consent from the Radiation 
Control Department Head, the Director of Jefferson Lab, and the Department of Energy. 

Calculations of the contribution to Jefferson Lab's annual radiation budget that would result 
from running under a broad variety of conditions typical of Hall~B operations indicate that 
the contribution from this experiment will be negligible. With this expectation, we have not 
carried out calculations for the specific running conditions of this experimental group. 

This expectation will be verified during the experiment by using the active monitors at the 
Jefferson Lab site boundary to keep up with the dose for the individual setups from Hall~B and 
the other Halls. If it appears that the radiation budget will be exceeded, the Radiation
Control Department will require a meeting with the experimenters and the Head of the 
Physics Division to determine if the experimental conditions are accurate, and to assess what 
actions may reduce the dose rates at the site boundary. If the site boundary dose approaches or
exceeds 10 mrem during any calendar year, the experimental program will stop until a resolution 
can be reached. 

\section{Radiation Hazards}
\indent

The following controls shall be used to prevent the unnecessary exposure of personnel and to 
comply with federal, state, and local regulations, as well as with Jefferson Lab and the 
experimenter's home institution policies. 

\subsection{From Beam in the Hall}

When the Hall status is Beam Permit, there are potentially lethal conditions present. Therefore, 
prior to going to Beam Permit, several actions will occur. Announcements will be made over the 
intercom system notifying personnel of a change in status from Restricted Access (free access 
to the Hall is allowed, with appropriate dosimetry and training) to Sweep Mode. All magnetic 
locks on exit doors will be activated. Persons trained to sweep the area will enter by keyed
access (Controlled Access) and search in all areas of the Hall to check for personnel. 

After the sweep, another announcement will be made, indicating a change to Power Permit, followed 
by Beam Permit. The lights will dim and Run-Safe boxes will indicate ``OPERATIONAL" and ``UNSAFE". 
IF YOU ARE IN THE HALL AT ANY TIME THAT THE RUN-SAFE BOXES INDICATE UNSAFE, IMMEDIATELY HIT THE 
BUTTON ON THE BOX. 

Controlled Area Radiation Monitors (CARMs) are located in strategic areas around the Hall and the 
Counting House to ensure that unsafe conditions do not occur in occupiable areas. 

\subsection{From Activation of Target and Beamline Components}

All radioactive materials brought to Jefferson Lab shall be identified to the Radiation Control 
Department. These materials include, but are not limited to, radioactive check sources (of any 
activity, exempt or non-exempt), previously used targets or radioactive beamline components, or 
previously used shielding or collimators. The Radiation Control Department inventories and tracks 
all radioactive materials onsite. The Radiation Control Department will survey all experimental 
setups before experiments begin as a baseline for future measurements. 

The Radiation Control Department will coordinate all movement of used targets, collimators, and 
shields. The Radiation Control Department will assess the radiation exposure conditions and will 
implement controls as necessary based on the radiological hazards. 

There shall be no local movement of activated target configurations without direct supervision by 
the Radiation Control Department. Remote movement of target configurations shall be permitted, 
providing the method of movement has been reviewed and approved by the Radiation Control Department. 

No work is to be performed on beamline components, which could result in dispersal of radioactive 
material (e.g., drilling, cutting, welding, etc.). Such activities must be conducted only with 
specific permission and control of the Radiation Control Department. 

\section{Incremental Shielding or Other Measures to be Taken to Reduce Radiation Hazards} 

None.

\section{Operations Procedures}

All experimenters must comply with experiment-specific administrative controls. These controls 
begin with the measures outlined in the experiment's Conduct of Operations Document, and also 
include, but are not limited to, Radiation Work Permits, Temporary Operational Safety Procedures, 
and Operational Safety Procedures, or any verbal instructions from the Radiation Control 
Department. A general access RWP is in place that governs access to Hall~B and the accelerator 
enclosure, which may be found in the Machine Control Center (MCC); it must be read and signed by 
all participants in the experiment. Any individual with a need to handle radioactive material at 
Jefferson Lab shall first complete Radiation Worker (RW~I) training.

There shall be adequate communication between the experimenter(s) and the Accelerator Crew Chief 
and/or Program Deputy to ensure that all power restrictions on the target are well known. Exceeding 
these power restrictions may lead to excessive and unnecessary contamination, activation, and 
personnel exposure. 

No scattering chamber or downstream component may be altered outside the scope of this RSAD 
without formal Radiation Control Department review. Alteration of these components (including 
the exit beamline itself) may result in increased radiation production from the Hall and a 
resultant increase in the site boundary dose. 

\section{Decommissioning and Decontamination of Radioactive Components}

Experimenters shall retain all targets and experimental equipment brought to Jefferson Lab for 
temporary use during the experiment. After sufficient decay of the radioactive target 
configurations, they shall be delivered to the experimenter's home institution for final 
disposition. All transportation shall be done in accordance with United States Department of 
Transportation Regulations (Title 49, Code of Federal Regulations) or International Air Transport
Association regulations. In the event that the experimenter's home institution cannot accept the 
radioactive material due to licensing requirements, the experimenter shall arrange for appropriate 
fund transfers for disposal of the material. Jefferson Lab cannot store indefinitely any radioactive 
targets or experimental equipment. 

{\bf The Radiation Control Department may be reached at any time through the
Accelerator Crew Chief (269-7050). }

\vspace{3.cm}
Approvals:

\vspace{1.cm}
\line(1,0){190}                    
\hspace{3cm}         \line(1,0){90}  

Radiation Control Department Head               \hspace{3cm}                  Date


\clearpage
\section{Appendix: Radiation Damage to the CLAS12 Electronics from the Beam Dump in the Hall~B 
Tagger Dipole Magnet Yoke}

While most of the CLAS12 electronics are located away from the beamline, the front-end electronics 
boards (FE boards) of the SVT and the MVT are in close proximity to the beamline, about 2~m to 3~m 
upstream of the target location. 

\subsection{Radiation Damage to the Standard Electronics Components in the Hall}

The experiment is not expected to produce significant levels of neutron radiation that may cause 
damage to the Hall~B electronics located on the Forward Carriage or on the Space Frame. The beam 
energies and the integrated luminosity of the experiment are in the same range as for the CLAS 
nuclear target experiments. Furthermore, all sensitive electrons that are mounted close to the 
beamline are upstream of the target where not much radiation is expected. Most of produced 
electromagnetic radiation (M{\"o}ller electrons) will be guided by 5~T solenoid field into tungsten 
shield pipe. Part of that radiation will be absorbed by shield part will be guided to the beam 
dump. Some of electronics modules mounted on Forward Tagger will be beyond the tungsten shield 
and will not see much radiation. 

\end{document}

