\section{Forward Tagger System}

The Forward Tagger system consists of three subsystems: an electromagnetic calorimeter 
(FT-Cal), a plastic scintillator hodoscope (FT-Hodo), and a MicroMegas-based tracker 
(FT-Trk). In the following, details are reported for each of the three subcomponents.

\subsection{Forward Tagger Calorimeter}

The Forward Tagger calorimeter (FT-Cal) consists of $332$ lead-tungstate (PbWO$_4$) crystals 
with avalanche photodiode (APD) readout and amplifiers enclosed inside a temperature-controlled 
enclosure. The crystals are arranged in a circular matrix positioned around the beamline. The 
system is located in proximity of magnets, in an area where the fringe field is of the order of 
a few hundred gauss. In order to operate the calorimeter, high voltage and low voltage are 
supplied to each channel. The high voltage is $<420$~V and $<50$~$\mu$A. The required low voltage 
is $\pm 5$~V for the preamplifier boards and 12~V for the Light Monitoring System. Constant 
temperature inside the enclosure is kept by running a coolant through the copper pipes that 
are integrated into the enclosure using a laboratory chiller. The cooling system should provide 
temperature stability at the level of $1^\circ$C. To avoid moisture build-up in the calorimeter 
enclosure, a steady flow of nitrogen gas is maintained and the temperature and humidity in the 
calorimeter enclosure are monitored with sensors interfaced to the CLAS12 Slow Controls system.

\subsubsection{Hazards} 

Hazards to personnel associated with this device are high voltage, which is supplied to 
the photosensors, and low voltage, which powers the calorimeter preamplifiers and Light 
Monitoring System. Hazards to the detector include cooling fluid leaks or condensation in 
the photosensors and preamplifier region, over-voltage to the photosensors, preamplifiers, 
or Light Monitoring System that could damage the related subsystem, and absence of cooling or 
cooling failure when low voltage is applied to the preamplifiers, which could lead to 
overheating of the preamplifiers themselves. To account for the presence of fringe magnetic 
fields from the CLAS12 magnets in the system location, no ferric materials are employed in 
the detector: a hazard may nevertheless arise during maintenance operations in case metallic 
tools are used and for people with cardiac pacemakers, other electrical medical devices, or 
metallic implants.

\subsubsection{Mitigations}

Mitigation of risks associated with FT-Cal operations are achieved in the following ways and
must be covered by ePAS Permit to work(s) (PTW).

\begin{itemize}
\item {Electrical shock: there is only a low level hazard for personnel related to the limited 
voltage/current range in use. Nevertheless, during maintenance periods,  HV and LV  power needs 
to be off before working on the calorimeter, cables disconnected, and lock and tags supplied;}
\item  {to avoid any damage to photosensors and preamplifiers, hardware  interlocks prevent 
any incorrect HV/LV settings; }
\item  {to avoid any damage to preamplifiers due to a possible coolant leakage, all cooling 
lines will be tested at high pressure and chiller parameters and temperatures monitored; 
hardware interlocks switch off the chiller in case of monitored parameters outside 
allowed range; }
\item {over-temperature causing damage to the preamplifiers will be avoided by continuously 
monitoring the status of the cooling and calorimeter temperature; interlocks will trigger HV/LV 
turn-off if the temperature exceeds the set values.}
\end{itemize}

\subsubsection{Electrical Hazard Mitigation (Personnel)}

High Voltage: high voltage up to 420~V is supplied to the photosensors (Hamamatsu S8664-1010 
Large Area Avalanche Photodiode or LAAPD). Mitigation: a very low current limit (50~$\mu$A) is 
set. All mechanical structures are properly grounded. HV distribution boards on the detector 
cannot be accessed during operation; during maintenance work, performed by trained personnel, 
the HV cables are disconnected from the power supply and the power supply locked and tagged.

Low Voltage: In order to power up the LAAPD preamplifier and Light Monitoring System, we use 
low voltage at $\pm$5~V and 12~V, respectively, with a maximum current of 4~A. Mitigation: 
voltage is low enough not to be a danger to personnel. All mechanical structures are properly 
grounded. All cables and connectors are certified for this rating. LV distribution boards on 
the detector cannot be accessed during operation; during maintenance work, performed by trained 
personnel, the LV cables are disconnected from the power supply and the power supply locked and 
tagged.

\subsubsection{Electrical Hazard Mitigation (Equipment)}

High Voltage: if high voltage is applied when the low voltage is turned off, the LAAPD 
preamplifiers may be damaged. Mitigation: the HV operation is interlocked to the LV settings, 
so that HV cannot be supplied if LV is off.

Over-voltage: applying HV and LV above certain values can damage the photosensors and 
preamplifiers. Mitigation: both HV and LV are monitored via the EPICS Slow Controls system; 
reading above predefined limits will automatically trigger the supplied voltage to be turned 
off.

\subsubsection{Other Hazard Mitigation}

Cooling fluid: leaks of the cooling fluid may cause damage to the calorimeter preamplifiers 
or nearby electronic components. Mitigations: during the design phase the cooling circuit 
path was chosen in order to minimize risks and, after the assembly, was tested at high 
pressure to verify the absence of leaks. During operation the temperature, level, and pressure 
of the liquid at the chiller output, as well as the temperatures of the calorimeter inlet and 
outlet lines, are monitored continuously via the EPICS Slow Controls system: any significant 
temperature variation (more than 1$^{\circ}$C) or sudden pressure variation must be investigated. 
The chiller operation is interlocked to these parameters so that variation outside appropriate 
limits will trigger the chiller being turned off.

Moisture: since the calorimeter is operated at 0$^{\circ}$C, moisture may build up in the 
system enclosure if the nitrogen gas flow is interrupted. Mitigation: the humidity inside the 
calorimeter enclosure is monitored via sensors; the operation of the LV and HV supplies are 
interlocked to the humidity readings so that, if the humidity exceeds a predefined threshold, 
both supplies will be automatically turned off.

Over-temperature: absence of cooling or cooling failure when LV is supplied to the preamplifiers 
may cause over-heating of the preamplifiers and other calorimeter components. Mitigation: cooling 
status and temperature inside the calorimeter enclosure are monitored and interlocked to the LV 
operation so that when the cooling is not in operation or temperatures exceed predefined limits, 
the LV is turned off.

Magnetic field: a hazard for personnel and equipment may arise if maintenance operations are 
performed while magnets are energized. Mitigation: the detector area is not accessible during 
regular CLAS12 operation; accessing the detector area implies the displacement of other CLAS12 
subsystems that requires the magnet to be turned off. Energized magnets are noted by red flashing 
beacons.

\vfil
\eject

\subsubsection{Responsible Personnel}

Individuals responsible for the FT-Cal system are (see Table~\ref{tb:ft-cal}):

\begin{table}[!htb]
\centering
\begin{tabular}{|c|c|c|c|c|} \hline
Name&Dept.&Phone&email&Comments \\ \hline
Expert on call& & 757-344-1848 && 1st contact \\ \hline
M. Battaglieri& INFN&757-269-7266&\href{mailto:battagli@jlab.org}{\nolinkurl{battagli@jlab.org}}&2nd contact \\ \hline
R. De Vita & JLab &757-269-5701&\href{mailto:devita@jlab.org}{\nolinkurl{devita@jlab.org}}& 3rd contact  \\ \hline
\end{tabular}
\caption{Personnel responsible for the CLAS12 FT-Cal system.} 
\label{tb:ft-cal}
\end{table}

\subsection{Forward Tagger Hodoscope}

The Forward Tagger Hodoscope (FT-Hodo) consists of $232$ plastic scintillator tiles (Eljen 
EJ 204) coupled to 6-m-long optical fibers with SiPM readout, preamplifier, and mezzanine 
and control electronic PCBs enclosed in an electronics crate. The system is located in the 
proximity of magnets, in an area where the fringe field is of the order of a few hundred gauss.

The FT-Hodo Light Monitoring System consists of a 420~nm (violet) peaked LED (Thorlabs M420F2), 
LED driver (LED D1B T-Cube), ten optical fibers, and eight cylindrical diffusers (Medilight). 
The LED and driver are located in the electronics rack and the optical diffusers are located 
in the plastic scintillator enclosure, four in each layer.

\subsubsection{Hazards} 

Hazards are electric shock if the electronics enclosure is opened without switching off the 
HV and LV, or exposure to non-ionizing UV radiation if the UV LED safety mitigations are not 
adhered to.

The LED is capable of producing high intensity UV light, which poses an eye and skin hazard. 

The SiPMs can be damaged if they are subjected to over-voltage or over-current and are sensitive 
to electrostatics.

To account for the presence of fringe magnetic fields from the CLAS12 magnets in the system 
location, no ferric materials are employed in the detector: a hazard may nevertheless arise 
during maintenance operations in case metallic tools are used and for people with cardiac 
pacemakers, other electrical medical devices, or metallic implants.

\subsubsection{Mitigations}
Any work on this system must be covered by ePAS Permit to work(s) (PTW).

Whenever any work has to be done on the Hodoscope, whether it will be opened or not, HV and 
LV cables are disconnected from the power supply and the power supply locked and tagged.

Junctions for the LED light are: LED to optical fiber (SMA connector), optical fiber into 
splitter box, splitter box to 11 output optical fibers, fibers to SiPM, and fibers illuminating 
the inside of the hodoscope. All of these junctions are completely sealed and light-tight. The 
fiber must not be disconnected from the LED source while the LED is operating. The splitter box 
or the hodoscope enclosure must not be opened while the LED is operating. The fibers should 
never be disconnected from the SiPMs while the LED is operating. 

The Light Monitoring System must not be turned on or left on when the electronics enclosure 
or plastic scintillator enclosure are opened. The LED should not be looked at directly - eye 
protection must be worn. Warning labels are applied to the LED enclosure and plastic 
scintillator enclosure.  During maintenance work, performed by trained personnel, the Light 
Monitoring System will be disconnected from power and locked and tagged.

A magnetic field hazard for personnel and equipment may arise if maintenance operations are 
performed while magnets are energized. Mitigation: the detector area is not accessible during 
regular CLAS12 operation; accessing the detector area implies the displacement of other CLAS12 
subsystem that require the magnet to be turned off. Energized magnets are noted by red flashing 
beacons.

\subsubsection{Responsible Personnel}

Individuals responsible for the FT-Hodo system are (see Table~\ref{tb:ft-hodo}):

\begin{table}[!htb]
\centering
 \begin{tabular}{|c|c|c|c|c|} \hline
 Name&University&email&Comments \\ \hline
 N. Zachariou & York &\href{mailto:nick.zachariou@york.ac.uk}{\nolinkurl{nick.zachariou@york.ac.uk}}& 1st contact  (general) \\ \hline
 D. Watts & York &\href{daniel.watts@york.ac.uk}{\nolinkurl{daniel.watts@york.ac.uk}}&2nd contact (general) \\ \hline
 D. Sokhan & Glasgow &\href{mailto:daria@jlab.org}{\nolinkurl{daria@jlab.org}}& 3rd contact (flasher) \\ \hline
  \hline
 \end{tabular}
 \caption{Personnel responsible for the CLAS12 FT-Hodo system.} 
\label{tb:ft-hodo}
 \end{table}
 

\subsection{Forward Tagger Tracker}

The Forward Tagger Tracker is composed of two double-sided Micromegas double-stage gaseous 
detectors. The system is located in the proximity of magnets, in an area where the fringe field 
is of the order of a few hundred gauss. The detectors are based on the resistive Micromegas 
technology that is also employed in the CLAS12 Central Micromegas tracker. Each detector 
consists of two planes with strips oriented along the $x$ and $y$ axes, respectively, separated 
by 10~mm. Each plane as 768 strips with 560~$\mu$m pitch. The FT tracker covers the polar angle 
region from 2.5$^\circ$ to 4.5$^\circ$ from the target. The fill gas is a mixture 
of 80\% argon, 10\% CF$_4$, and 10\% isobutane. Even though isobutane is a flammable gas, the 
amount of gas in use at any given time is well within the range of a class-0 gaseous device.

The FT-Tracker is powered by high voltages up to 2000~V. The current limit for all three HV is 
extremely low ($< 1$~mA). The detectors are read out through 1.5-m-long, low-capacitance flex 
cables by Front-End Units (FEU). These electronic cards contain the customized DREAM ASICs in 
order to sample the detector signal and a Flash-ADC to digitize it and send it to the network. 
The FEUs are placed inside customized crates, installed on the outer case of the FT calorimeter. 
They are powered through low voltage, and kept within a 40$^\circ$C to 60$^\circ$C temperature 
range using a simple set of fans and tubing.

All hazards and mitigation options for the FT tracker are the same as for the CLAS12 Central 
Micromegas tracker (FMT and BMT). Even though the shapes of the detectors vary, they are almost 
identical in principle. 

\subsection{Hazards}

Hazards to personnel include the use of flammable gas, high voltage, and the low voltage that 
powers the readout electronics. Hazards to the detector include mechanical damage, gas leaks, 
and gas over-pressure. Hazards concerning the Front-End Units include: wrong LV settings that 
could damage the FEUs and absence of cooling or cooling failure that could overheat the cards. 
To account for the presence of fringe magnetic fields from the CLAS12 magnets in the system 
location, no ferric materials are employed in the detector: a hazard may nevertheless arise 
during maintenance operations in case metallic tools are used and for people with cardiac 
pacemakers, other electrical medical devices, or metallic implants.

\subsection{Mitigations}

Mitigation of risks associated with then FT-Trk operations are achieved in the following ways
and must be covered by ePAS Permit to work(s) (PTW).

\begin{itemize}
\item{ the limited amount of flammable gas (10$\%$ isobutane over the whole gas in the detector)  
makes this a class-0 gas system. Nevertheless, the supply and exhaust are located outside 
Hall~B, all flows monitored, and hardware interlocks are used to avoid leaks and over-pressure;}
\item{ the low voltage/current range represents  a low level electrical  hazard for personnel. 
Nevertheless, during maintenance the HV and LV will be powered-off, cables disconnected, and 
locks and tags supplied; }
\item{ Interlocks on the gas supply will mitigate possible leaks and over-pressure conditions;}
\item{ continuous LV monitoring in conjunction with hardware interlocks will reduce possible 
issues with wrong settings. FT-Trk temperature monitoring will provide a fast LV turn-off in 
case of cooling failures or over-temperature conditions. }
\end{itemize}

\subsubsection{Fire Hazard Mitigation (Equipment \\ and Personnel)}

Use of flammable gas: the FT-Trk detectors use 10\% isobutane, which is flammable. 
However the amount of isobutane in the system is very limited. The gas supply and exhaust 
are located outside Hall~B. The total combustion energy is equivalent to less than 4~g of 
hydrogen, which makes it a class-0 gas system (class-1 starts at 600~g). 

General fire hazards and procedures for dealing with these are covered by JLab emergency 
management procedures. The JLab Fire Protection Manager (Tim Minga) can be contacted at 
Office: 757-269-7310, Cell: 757-371-1687.

\subsubsection{Electrical Hazard Mitigation (Personnel)}

High Voltage: high voltages up to 2000~V are used routinely for the detectors. Mitigation: very 
low current limit (10~$\mu$A) is set. All mechanical structures are properly grounded.

Low Voltage: In order to power up the front-end electronics, we use low voltage at 6~V with 15~A 
per crate. Mitigation: voltage is low enough not to be a danger to personnel. All mechanical 
structures are properly grounded. All cables and connectors are certified for this rating.

\subsubsection{Other Hazard Mitigation}

Detectors: gas over-pressure, gas leaks, mechanical damage. Mitigation: gas control system with 
over-pressure and leak limit detection by flow meter (limit = 0.4~l/hr). The FT tracker detectors 
can be dismounted and repaired in case of accidental damage to the drift electrode. Since they are 
tightly stacked, only one such electrode is exposed and at risk, the rest of the stack is protected 
by the first detector.

Electronics: wrong LV settings, absence of cooling or cooling failure. Mitigation: Slow Controls 
read-back of LV setting before turning on the front-end electronics. Cooling is also checked by 
the Slow Controls system, and is interlocked so that the electronics cannot be turned on when the 
cooling is off. Also, temperature sensors are present on the front-end cards and are directly 
interlocked so that if the temperature goes beyond a predefined threshold, the cards are shut-down 
automatically.

Magnetic field: a hazard for personnel and equipment may arise if maintenance operations are 
performed while magnets are energized. Mitigation: the detector area is not accessible during 
regular CLAS12 operation; accessing the detector area implies the displacement of other CLAS12 
subsystems that require the magnet to be turned off. Energized magnets are noted by red flashing 
beacons.

\subsubsection{Responsible Personnel}

Individuals responsible for the FT-Trk system are (see Table~\ref{tb:ft-trk}):

\begin{table}[!htb]
\centering
\begin{tabular}{|c|c|c|c|c|} \hline
Name          & Dept. & Phone & email & Comments \\ \hline
Expert on call&       &       &       & 1st contact \\ \hline
R. Paremuzyan & JLab  &757-541-7539&\href{mailto:rafopar@jlab.org}{\nolinkurl{rafopar@jlab.org}}&2nd contact \\ \hline
Y. Gotra      & JLab  &757-269-5571&\href{mailto:gotra@jlab.org}{\nolinkurl{gotra@jlab.org}}&3rd contact \\ \hline
R. De Vita    & JLab  &757-269-7266&\href{mailto:devita@jlab.org}{\nolinkurl{devita@jlab.org}}&4th contact \\ \hline
\end{tabular}
\caption{Personnel responsible for the CLAS12 FT-Trk system.} 
\label{tb:ft-trk}
\end{table}
