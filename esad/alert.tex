\section{ALERT Detector}

The ALERT detector system consists of two subsystems: a drift chamber (AHDC) and a time-of-flight
scintillator detector (ATOF). In the following, details are reported for these subcomponents.

\subsection{ALERT Hyperbolic Drift Chamber}

The Forward Tagger calorimeter (FT-Cal) consists of $332$ lead-tungstate (PbWO$_4$) crystals 
with avalanche photodiode (APD) readout and amplifiers enclosed inside a temperature-controlled 
enclosure. The crystals are arranged in a circular matrix positioned around the beamline. The 
system is located in proximity of magnets, in an area where the fringe field is of the order of 
a few hundred gauss. In order to operate the calorimeter, high voltage and low voltage are 
supplied to each channel. The high voltage is $<420$~V and $<50$~$\mu$A. The required low voltage 
is $\pm 5$~V for the preamplifier boards and 12~V for the Light Monitoring System. Constant 
temperature inside the enclosure is kept by running a coolant through the copper pipes that 
are integrated into the enclosure using a laboratory chiller. The cooling system should provide 
temperature stability at the level of $1^\circ$C. To avoid moisture build-up in the calorimeter 
enclosure, a steady flow of nitrogen gas is maintained and the temperature and humidity in the 
calorimeter enclosure are monitored with sensors interfaced to the CLAS12 Slow Controls system.

\subsection{Hazards} 

Hazards to personnel include the high voltage supplied to the wires and the low voltage that 
powers the on-chamber pre-amplifiers. Hazards to the drift chambers themselves include damage 
to the gas windows should the pressure deviate more than a few psi from atmospheric. 

\subsection{Mitigations}

Electrical hazards:
\begin{itemize}
\item High Voltage: high voltages up to 2000~V are used routinely for all detectors. Mitigation: 
very low current limits (40~$\mu$A) are set. All mechanical structures are properly grounded.  
There are possible electrical hazards if a malfunctioning HV board is replaced. The associated
ePAS concluded that the risk is low, but any work on this system must be covered by ePAS Permit
to work(s) (PTW).

\item Low Voltage: In order to power up the on-chamber electronics, we use low voltage at 7~V 
with 50~A per supply (1 supply per chamber). Mitigation: voltage is low enough not to be a 
danger to personnel. All mechanical structures are properly grounded. All cables and connectors 
are certified for this rating and shielded. To protect against possible over-heating of the 
on-chamber pre-amplifier boards, each individual conductor (positive and neutral return) is 
fused; with the fuses located in a fuse panel with a red LED signaling a blown fuse. If a 
fuse is removed and/or replaced there is no risk to personnel because of the low voltage.

\end{itemize}

Gas system hazards:
\begin{itemize}
\item Personnel: because most of the system operates very close to atmospheric pressure there 
is no hazard to personnel in the hall due to pressure. The gas is non-toxic and non-flammable.  
Because of the large volume of the hall and the location of the chambers in the main open area 
of the hall, there is no ODH hazard to personnel.
\item Detectors: there is a potential danger to the chamber gas windows if the pressure in the 
chamber differs from atmospheric by one psi. This is mitigated during standard operation by our 
pressure-difference control system with fail-safe over-pressure and under-pressure bubblers 
providing an additional level of safety.
\end{itemize}

\subsection{Responsible Personnel}

Individuals responsible for the DC system are (see Table~\ref{tb:dc}):

\begin{table}[!htb]
\centering
\begin{tabular}{|c|c|c|c|c|} \hline
Name            & Dept.  & Phone        & email&Comments \\ \hline
Expert on call  &        & 757-748-5048 &       & 1st contact \\ \hline
Gabriel Charles & Hall~B & 757-746-3395 & \href{mailto:hauenst@jlab.org}{\nolinkurl{hauenst@jlab.org}} &  2nd contact\\ Raphael Dupre   & Hall~B & 757-746-3395 & \href{mailto:hauenst@jlab.org}{\nolinkurl{hauenst@jlab.org}} &  2nd contact\\ \hline
 \end{tabular}
\caption{Personnel responsible for the ALERT AHDC system.} 
\label{tb:ahdc}
\end{table}

\subsection{ALERT Time-of-Flight System}

The Forward Tagger Hodoscope (FT-Hodo) consists of $232$ plastic scintillator tiles (Eljen 
EJ 204) coupled to 6-m-long optical fibers with SiPM readout, preamplifier, and mezzanine 
and control electronic PCBs enclosed in an electronics crate. The system is located in the 
proximity of magnets, in an area where the fringe field is of the order of a few hundred gauss.

The FT-Hodo Light Monitoring System consists of a 420~nm (violet) peaked LED (Thorlabs M420F2), 
LED driver (LED D1B T-Cube), ten optical fibers, and eight cylindrical diffusers (Medilight). 
The LED and driver are located in the electronics rack and the optical diffusers are located 
in the plastic scintillator enclosure, four in each layer.

\subsection{Hazards} 

There are two hazards associated with the FTOF system related to i) the high voltage (HV)
system used to energize the counter PMTs and ii) access to the counters during testing 
operations.

The HV power supplies for each FTOF sector are either CAEN 1527 or 4527 mainframes outfitted
with negative polarity 24-channel A1535N modules. The typical settings
for each channel are: $V=-2000$~V, $I=350$~$\mu$A. These supplies are located on the north 
and south sides of each level of the Forward Carriage behind each sector of counters. There 
are two hazards associated with the HV system when energized that must be mitigated. The 
first is the electrical hazard and the second is the potential damage to PMTs if a light 
leak is introduced in the counter wrapping material when the PMT is energized. All HV channel
supply currents are monitored by the EPICS Slow Controls system.

The panel-1b and panel-1a counters are positioned between the LTCC and PCAL detectors on 
the Forward Carriage. Therefore they are not accessible for hands-on testing. However, 
the panel-2 counters are accessible for hands-on testing when the Forward Carriage is 
pulled back into its maintenance position. The panel-2 counters in the S1, S2, S3, and S4 
positions can then be accessed by manlift and the panel-2 counters in the S5 and S6 
positions can be accessed by either manlift or ladders. When testing the panel-2 counters 
in such an operation there are fall hazards that must be mitigated.

\subsection{Mitigations}

The electrical hazard associated with the HV system would be to receive an electrical 
shock. However, the design of the HV system for the FTOF is such that the chance to 
receive an electrical shock is minimal. The electrical hazards are mitigated by the use 
of properly rated RG-59 cables that are terminated at the voltage divider end and the HV 
supply end. As well, the HV supplies are grounded to their electronics racks. The bigger 
issue would be damage to a PMT if improper contact with the counter surface were to occur 
that introduced a sizable light leak in the counter wrapping. However, the hazards in such 
a situation are minimal in that the HV system is designed to shutdown any channels that 
show an over-current condition, thereby protecting the system hardware. 

Only authorized FTOF system personnel are allowed to work on the counters during hands-on
testing when the Hall~B configuration allows such work. For these individuals using ladders 
or manlifts, they are required to have all appropriate training including manlift and
harness training, ladder training, and fall protection training as required by ePAS Permit
to work(s) (PTW).  All work is carried out in conjunction with input from the FTOF Group
Leader and the Hall~B Work Coordinator.

\subsection{Responsible Personnel}

Individuals responsible for the FTOF system are (see Table~\ref{tb:ftof}):

\begin{table}[!htb]
\centering
\begin{tabular}{|c|c|c|c|c|} \hline
Name              & Dept.  & Phone        & email & Comments \\ \hline
FTOF/CTOF on call & Hall B & 757-344-7204 &       & 1st contact \\ \hline
Whitney Armstrong & Hall B & 757-269-5586 & \href{mailto:carman@jlab.org}{\nolinkurl{carman@jlab.org}} & 2nd contact \\ \hline
Raphael Dupre     & Hall B & 757-269-5586 & \href{mailto:carman@jlab.org}{\nolinkurl{carman@jlab.org}} & 2nd contact \\ \hline
\end{tabular}
\caption{Personnel responsible for the ALERT ATOF system.} 
\label{tb:atof}
\end{table}
