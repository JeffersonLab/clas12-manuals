\section{ALERT Detector}

The ALERT detector system consists of two subsystems: a drift chamber (AHDC) and a time-of-flight
scintillator detector (ATOF). In the following, details are reported for these subcomponents.

\subsection{ALERT Hyperbolic Drift Chamber}

The ALERT Hyperbolic Drift Chamber (AHDC) is a cylindrical drift chamber placed along the beamline
within the CLAS12 5~T solenoid. It consists of 3026 wires arranged in 8 layers of sense wires (totalling
576 sense wires) arranged along the beamline alternating with $\pm$10$^\circ$ stereo angle. Between
each layer of sense wires are strung layers of field wires. The gas system supplies mixed, clean,
pressure-controlled 90\% He/10\% C$_4$H$_{10}$ gas. The chambers are connected via 2-m-long, 64-channel
multi-conductor signal cables to front-end readout boards located on the ALERT detector installation
cart. The 512-channel readout boards each contain 8 64-channel DREAM ASIC preamplifier/shapers. The
high voltage (HV) is supplied to the on-chamber HV adapter boards from the local HV crate. The sense
wires are at positive HV and the field and guard wires are grounded.

\subsection{Hazards} 

There are four hazards to personnel associated with the ALERT AHDC system related to i) the high
voltage (HV) system supplied to the wires, ii) the LV system that powers the pre-amplifiers,
iii) the solenoid magnetic field, and iv) access to the detector during testing operations. The
hazard to the AHDC hardware includes damage to the gas enclosure should the pressure deviate more
than a few psi from atmospheric. 

\subsection{Mitigations}

Electrical hazards:
\begin{itemize}
\item High Voltage: high voltages up to 2000~V are used routinely for all detectors. Mitigation: 
very low current limits (40~$\mu$A) are set. All mechanical structures are properly grounded.  
There are possible electrical hazards if a malfunctioning HV board is replaced. The associated
ePAS concluded that the risk is low, but any work on this system must be covered by ePAS Permit
to Work(s) (PTW).

\item Low Voltage: In order to power up the pre-amplifiers, we use low voltage at $<10$~V with
50~A per supply. Mitigation: power is low enough not to be a danger to personnel. All mechanical
structures are properly grounded. All cables and connectors are certified for this rating and
shielded. The amplifiers are housed within the JLab-design front-end electronics units are are
suitably protected from over-heating, over-current, and over-voltage.

\end{itemize}

Gas system hazards:
\begin{itemize}
\item Personnel: because most of the system operates very close to atmospheric pressure there 
is no hazard to personnel in the hall due to pressure. The gas is non-toxic and flammable.  
Because of the large volume of the hall and the location of the chambers in the main open area 
of the hall, there is no ODH hazard to personnel.
\item Detectors: there is a potential danger to the ALERT detector gas enclosure if the pressure
in the detector differs from atmospheric by one psi. This is mitigated during standard operation by our 
pressure-difference control system with fail-safe over-pressure and under-pressure bubblers 
providing an additional level of safety.
\end{itemize}

The ALERT detector is positioned in the magnetic field of the CLAS12 solenoid. When the solenoid is
energized to its full nominal current, the central field strength is 5~T and the field strength at
upstream end of the detector is at the level of 1~kG. This field level presents a possible hazard to
both personnel and to ALERT detectors (as well as the other detectors in located about the solenoid).
As such no service work on the ALERT detector is to take place when the solenoid is energized.

During testing or repairs with the solenoid off, it is possible to access the front-end electronics
racks and the upstream end of the ALERT detector. If there is need for the use of ladders or platforms,
there are fall hazards that must be mitigated. Any work on this system must be covered by ePAS Permit to
Work(s) (PTW).

\subsection{Responsible Personnel}

Individuals responsible for the ALERT AHDC system are (see Table~\ref{tb:ahdc}):

\begin{table}[!htb]
\centering
\begin{tabular}{|c|c|c|c|} \hline
Name           & Dept.  & Contact &Comments \\ \hline
Expert on call &        & 757-329-4844 & 1st contact \\ \hline
G. Charles     & Orsay & \href{mailto:gabriel.charles@ijclab.in2p3.fr}{\nolinkurl{gabriel.charles@ijclab.in2p3.fr}} &  2nd contact\\ \hline
R. Dupr\'e     & Orsay & \href{mailto:raphael.dupre@ijclab.in2p3.fr}{\nolinkurl{raphael.dupre@ijclab.in2p3.fr}} &  3rd contact\\ \hline
 \end{tabular}
\caption{Personnel responsible for the ALERT AHDC system.} 
\label{tb:ahdc}
\end{table}

\subsection{ALERT Time-of-Flight System}

The ALERT Time-of-Flight detector (ATOF) consists of 15 modules mounted parallel to the beamline
that connect to a support framework radially outward of the AHDC. Each ATOF module includes a single
thin scintillator bar along the length of the module on its inside face backed by 10 thicker scintillator
wedges. The bars and wedges are read out by silicon photomultipliers (SiPMs) glued to the scintillators.
The ATOF front-end electronics for each module consists of a hybrid system incorporating both JLab-designed
Petrioc2A and NALU Electronics ASOC boards. These boards are used to shape the signals and provide both
energy (ADC) and timing (TDC) information. The front-end electronics located on the ALERT installation
cart also contact signal distribution boards for power, clock, and timing signals.

\subsection{Hazards}

There are three hazards associated with the ATOF system related to i) the low voltage system used
to energize the ATOF SiPMs, ii) the solenoid magnetic field, and iii) access to the counters during
testing operations.

\subsection{Mitigations}

The low voltage power supply for the ATOF counters has typical settings for each channel are: $V=100$~V.
This supply is located in the electronics racks attached to the ALERT installation cart. There are two
hazards associated with the LV system when energized that must be mitigated. The first is the electrical
hazard and the second is the potential damage to SiPMs if a light leak is introduced in the counter
wrapping material when the PMT is energized.

The electrical hazard associated with the LV system would be to receive an electrical shock. However, the
design of the LV system for the ATOF is such that the chance to receive an electrical shock is minimal.
The electrical hazards are mitigated by the use of properly rated power cables that are terminated at the
SiPM end and the power supply end. As well, the supplies are grounded to their electronics racks. The bigger
issue would be damage to a SiPM if improper contact with the counter surface were to occur that introduced 
a sizable light leak in the counter wrapping. However, the hazards in such a situation are minimal in that
the system is designed to shutdown any channels that show an over-current condition, thereby protecting the
system hardware. All supply currents are monitored by the EPICS Slow Controls system.

The ALERT detector is positioned in the magnetic field of the CLAS12 solenoid. When the solenoid is
energized to its full nominal current, the central field strength is 5~T and the field strength at
upstream end of the detector is at the level of 1~kG. This field level presents a possible hazard to
both personnel and to ALERT detectors (as well as the other detectors in located about the solenoid).
As such no service work on the ALERT detector is to take place when the solenoid is energized.

During testing or repairs with the solenoid off, it is possible to access the front-end electronics
racks and the upstream end of the ALERT detector. If there is need for the use of ladders or platforms,
there are fall hazards that must be mitigated. Any work on this system must be covered by ePAS Permit to
Work(s) (PTW).

\subsection{Responsible Personnel}

Individuals responsible for the ALERT ATOF system are (see Table~\ref{tb:atof}):

\begin{table}[!htb]
\centering
\begin{tabular}{|c|c|c|c|} \hline
Name           & Dept. & Contact & Comments \\ \hline
Expert on call &       & 757-329-4844 & 1st contact \\ \hline
W. Armstrong   & ANL   & \href{mailto:warmstrong@anl.gov}{\nolinkurl{warmstrong@anl.gov}} & 2nd contact \\ \hline
R. Dupr\'e     & Orsay & \href{mailto:raphael.dupre@ijclab.in2p3.fr}{\nolinkurl{raphael.dupre@ijclab.in2p3.fr}} & 3rd contact\\ \hline
\end{tabular}
\caption{Personnel responsible for the ALERT ATOF system.} 
\label{tb:atof}
\end{table}
