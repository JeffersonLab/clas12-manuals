\section{Superconducting Solenoid Magnet}

The CLAS12 Solenoid magnet provides the magnetic field for the tracking of charged 
particles and suppression of low energy electron background. It hosts several detector 
packages including the Central Vertex Tracker (SVT and MVT), the Central Time-of-Flight, 
and the Central Neutron Detector. They all are located in the 780-mm-diameter warm bore. 
The solenoid has four main coils and one shield coil. The solenoid produces a magnetic 
field of 5~T when powered at 2416~A. The magnet has an overall inductance of 5.89~H and 
stored energy of 17.2~MJ.

\subsection{Hazards} 

The hazards of the Solenoid magnet include the following:

\begin{itemize}
\item Electrical hazard
\item Cryogenic hazard
\item Vacuum hazard
\item Magnetic field
\item Stored energy
\end{itemize}

\subsection{Mitigations}
Any work on this system must be covered by ePAS Permit to work(s) (PTW).

\subsubsection{Electrical Hazard}

The power supply for the Solenoid operates with input voltages of 120~VAC and 480~VAC and 
is interlocked to a current limit of 2450~A. Maintenance and servicing of the power supply 
can only be conducted by ``Qualified Electrical Workers''. Additional information can be 
found in the ePAS and procedures for the Hall~B solenoid magnet. During normal operation, connections at 
the power supply are made inside the cabinet that has interlocked doors. Insulated cables 
carrying current to the magnet are routed with cable trays with all exposed leads and 
terminations covered by non-conductive or expanded metal enclosures. During a fast dump or 
quench, high voltage spikes may be induced on current leads and voltage taps. The leads from
the voltage tap wires connect to the control system wiring through current limiting resistors 
to reduce any current-voltage combination to within the class-1 Electrical Classification of 
the EHS\&Q Manual.

\subsubsection{Cryogenic Hazard}

Nitrogen and helium are two types of cryogens used to keep the coils superconducting. The 
total volume of liquid helium and liquid nitrogen in Hall~B is less than 900~liters and 
130~liters, respectively. Proper insulation is installed on all piping accessible to 
personnel. In the event of a quench or loss of insulating vacuum event, relief valves on 
the helium and nitrogen circuits vent generated gas to the hall. In case of such an event, 
Hall~B remains ODH-0. In case of a power outage, the hall ODH rating would go up to ODH-2 
after five hours. Appropriate ODH signs are posted at all entrances to the hall and an 
oxygen monitoring system is installed in the hall and operational.

\subsubsection{Vacuum Hazard}

The purpose of the vacuum system is to provide $10^{-5}$~Torr or better thermal insulating 
vacuum to four superconducting coils and one cryogenic distribution box. After liquid helium 
is introduced into the coils, a Loss of Vacuum (LOV) event with a full air inrush can lead to
very high heat transfer to the helium and nitrogen circuits with a resulting phase change in 
the liquid helium and nitrogen and potential high pressure expulsion from the system. In the 
event of an LOV event, relief valves on the helium and nitrogen circuits vent generated
gas to the hall.

\subsubsection{Magnetic Field}

When powered up to 2416~A, the Solenoid can generate up to 5~T field in the center of the 
magnet and up to 1~kG in the zones that extend beyond the magnet boundaries. The 5~G 
boundary restricting access by personnel with surgical implants and bioelectric devices, 
the 200~G crane boundary, and the 600~G whole body boundary were found and recorded during 
the commissioning of the magnet. These contours are be marked up and appropriate signage 
posted. Strong magnetic fields will attract loose ferromagnetic objects, possibly injuring 
body parts or striking fragile components. Prior to energizing the magnet, a sweep of the
surrounding area must be performed for any loose magnetic objects. All personnel entering 
the 600~G area will also be trained to remove ferromagnetic objects from themselves. To 
prevent personnel with surgical implants and bioelectric devices from entering the 5~G
boundary, lighted warning signs are placed at the doors of the hall when the Solenoid is 
energized, and flashing red beacons and personnel barricades are installed at the actual 5~G 
contour.

\subsubsection{Stored Energy}

At 2416~A, the total energy stored in the magnet is about 17.2~MJ. Upon sudden loss of hall 
electrical power or quench or LOV, the energy is dumped into a dump resistor.

\subsection{Responsible Personnel}

Individuals responsible for the CLAS12 solenoid system are (see Table~\ref{tb:solenoid}):

\begin{table}[!htb]
\centering
\begin{tabular}{|c|c|c|c|c|} \hline
Name&Dept.&Phone&email&Comments \\ \hline
Engineering on call& Hall~B & 757-748-5048 &$-$& 1st contact \\ \hline
B. Miller          & Hall~B & 757-269-7867 &\href{mailto:miller@jlab.org}{\nolinkurl{miller@jlab.org}}&2nd contact \\ \hline
K. Bruhwel         & Hall~B & 757-269-5577 &\href{mailto:}{\nolinkurl{bruhwel@jlab.org}}&3rd contact \\ \hline
\end{tabular}
\caption{Personnel responsible for the CLAS12 solenoid magnet system.} 
\label{tb:solenoid}
\end{table}

