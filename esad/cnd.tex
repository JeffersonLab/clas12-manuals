\section{Central Neutron Detector}

The Central Neutron Detector (CND) is the outer-most detector of the CLAS12 Central
Detector. The CND is a barrel of plastic scintillator bars of trapezoidal shape, all 
with their long sides parallel to the beam direction. 

The light emitted by the scintillators of the CND is read out only at the upstream end of each 
bar with a Hamamatsu R10533 photomultiplier placed in the low-field region of the solenoid 
and connected to the bar by a $\sim 1.5$-m-long bent light guide; the downstream end of each 
bar is connected via a ``U-turn'' light guide to the neighboring paddle. In this way, the 
light emitted at the downstream end of each scintillator is fed through its neighboring 
paddle and read out by the PMT connected to its end. Each PMT is encased in a cylindrical 
magnetic shield made up by a 1-mm-thick layer of mu-metal and a 5-mm-thick layer of mild steel. 

The CND is composed of 48 azimuthal segments and 3 layers in the radial direction, for a total 
of 144 scintillator bars, 144 PMTs, 72 U-turn light guides, and 144 bent light guides. 

In order to operate the PMTs, high voltages (typically in the range of 1500~V) are provided  
by a multi-channel CAEN SY527 power supply. The signal of each PMT is sent to an active splitter. 
The three splitter modules used for the CND were originally developed by IPN Orsay for the G0 
experiment (Hall~C, Jefferson Lab). Each module is an active 64-channel splitter with unity gain 
so there is no loss of amplitude. The 64 SMA inputs are placed in the back panel. In the front 
panel there are 8 8-channel output connectors (DMCH) for the timing signals and 4 16-channel 
output connectors (FASTBUS) for the charge signals. The charge signal is sent from the splitter 
to the flash-ADC (250 VXS, 16 channels/board, made and owned by JLab). The timing signal from 
the splitter is sent to a constant fraction discriminator (CFD) GAN'ELEC FCC8, developed for the 
TAPS Collaboration. The module is an 8-channel CAMAC unit with LEMO 00 input connectors and 2x 
8-pin output connectors in differential ELC. The threshold can be set for each channel 
individually and no time-walk adjustment is required for the module. The discriminated timing 
signal then goes to the TDC (CAEN VX1290A, 32 channels/board, 25 ps/channel resolution). In total, 
the read-out of the CND includes 3 splitter modules, 19 CFD modules, 5 TDC boards, and 8 ADC 
boards. 

\subsection{Hazards} 

\subsubsection{Electrical Hazard}

The electrical hazard to personnel can come from the high voltage that powers the PMTs, which 
need about 1500~V to function. 

\subsubsection{Magnetic Field Hazard}

The strong magnetic field of the solenoid (5~T) represents a hazard for all detectors of the
CND.

\subsection{Mitigations}
Any work on this system must be covered by ePAS Permit to work(s) (PTW).

\subsubsection{Electrical Hazard Mitigations} 

The maximum current provided by the HV distribution boards is quite low (3~mA). All mechanical 
structures are properly grounded. The HV boards must not be accessed during operation; during 
maintenance work, performed by trained personnel, the HV is turned off, cables are disconnected 
from the power supply and the power supply is turned off. 

\subsubsection{Magnetic Field Hazard Mitigations}

Whenever any work has to be done on the CND, the magnetic field of the solenoid must be turned 
off. After any sort of maintenance work is done on the CND, the area must be inspected and all 
ferromagnetic tools and equipment must be removed before the field is ramped up again. Also, 
before the field can be turned on the PMT housings and magnetic shields should be thoroughly 
inspected to make sure that they are properly secured and that there are no loose parts. 

\subsection{Responsible Personnel}

Individuals responsible for the CND system are:

\begin{table}[!htb]
\centering
\begin{tabular}{|c|c|c|c|c|} \hline
Name&Dept.&Phone&email&Comments \\ \hline
Expert on call& &&& 1st contact \\ \hline
S. Niccolai& IPN Orsay&+33 6 24 81 67 78&\href{mailto:silvia@jlab.org}{\nolinkurl{silvia@jlab.org}}& 2nd contact \\ \hline
D. Sokhan & Glasgow & + 44 7949 175725 &\href{mailto:daria@jlab.org}{\nolinkurl{daria@jlab.org}} & 3rd contact  \\ \hline
D.S. Carman & JLab & 757-269-5586 & \href{mailto:carman@jlab.org}{\nolinkurl{carman@jlab.org}} & JLab contact \\ \hline
\end{tabular}
\caption{Personnel responsible for the CLAS12 CND system.} 
\label{tb:cnd}
\end{table}

