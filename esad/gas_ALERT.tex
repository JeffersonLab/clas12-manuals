\section{Hall~B Gas System}

The Hall~B gas systems supply gas to the following detectors at the indicated pressures 
in Hall~B:
\begin{enumerate}
\item Forward Drift Chambers 
\begin{description}
\item[-] 10\% CO$_2$ in Argon at $0.075$~inch wc
\item[-] N$_2$ purge for external HV components at atmospheric pressure
\end{description}
\item LTCC 
\begin{description}
\item[-] C$_4$F$_{10}$ at $1.0 - 4.0$~inch wc
\item[-] N$_2$ purge for C$_4$F$_{10}$ recovery at $1.0-4.0$~inch wc
\end{description}
\item ALERT
  \begin{description}
    \item[-] 80\% He + 20\% CO$_2$ at 1060 mbar
\end{description}
%\item MVT 
%\begin{description}
%\item[-] 10\% C$_4$H$_{10}$ in Argon - 30~psi
%\item[-] 10\% CF$_4$ 10\% C$_4$H$_{10}$ in Argon - 30~psi 
%\end{description}
\item RICH 
\begin{description}
\item[-] N$_2$ purge for aerogel at atmospheric pressure
\item[-] Air cooling purge supply for enclosed electronics package at 5-55~psi, discharges to atmosphere
\item[-] Air compressor output at 112~psi 
\end{description}
\item HTCC 
\begin{description}
\item[-] CO$_2$ purge at $0.150$~inch wc 
\end{description}
\item FT 
\begin{description}
%\item[-] 10\% CF$_4$ + 10\% C$_4$H$_{10}$ in Argon - 30~psi
\item[-] N$_2$ purge for calorimeter at atmospheric pressure 
\end{description}
%\item SVT
%\begin{description}
%\item[-] Dry air purge at atmospheric pressure 
%\end{description}
\end{enumerate}

The Hall~B Gas Controls consists of a National Instruments cRIO based controls system 
that supplies and monitors gas flow to the three baseline detectors (DC, HTCC, 
LTCC). Additionally, it also controls and monitors gas supply to three non-baseline 
detectors (ALERT, FT, RICH). 

The system consists of four stations strategically located in the hall and Gas Shed that 
are linked via the Slow Controls network. Each station consists of a cRIO controls chassis, 
a custom interface chassis, and a touch screen monitor. All gas system instrumentation 
equipment (transducers, mass flow controllers, valve drivers, etc.) receive operational 
power from supplies that are internal to the custom interface chassis. 

The main controls interface for the system is located in the Gas Shed, where all functions 
of the system are controlled. All system chassis are electrically grounded to the racks, 
and each contains an overcurrent protection fuse. 

\subsection{Hazards} 

The following cryogens are used at the 96B Gas Shed at the following pressures:
\begin{itemize}
\item Liquid Argon - $175 - 200$~psi
\item Liquid Nitrogen - $45$~psi
\item Liquid CO$_2$ - $160 - 300$~psi
\end{itemize}

\noindent
The following gases are produced from cryogen boil off:

\begin{itemize}
\item Ar - $160 - 200$~psi
\item CO$_2$ - $160 - 200$~psi
\item N$_2$ - $35$~psi
\end{itemize}

\noindent
The following gases are used at the 96B Gas Shed at the following pressures:

\begin{itemize}
\item Ar - $40 - 200$~psi
\item CO$_2$ - $15-200$~psi
\item CF$_4$ - $40$~psi
\item C$_4$F$_{10}$ - $4 - 50$~psi
\item C$_4$H$_{10}$ - $40$~psi
\item N$_2$ - $35$~psi
\end{itemize}

\noindent
The following gas mixtures are produced at the following pressures:

\begin{itemize}
\item 10\% CO$_2$ in Argon - $100$~psi
\item 10\% C$_4$H$_{10}$ in Argon - $15$~psi (not used for RG-L)
\item 10\% CF$_4$ $10\%$ C$_4$H$_{10}$ in Argon - $15$~psi (not used for RG-L)
\end{itemize}

\noindent
The ALERT gas mixture is non-flammable:

\begin{itemize}
\item 80\% He 20\% CO$_2$ - 1060 mbar
\end{itemize}

\noindent
The following gases and gas mixtures are sent to Hall~B at the following pressures:

\begin{itemize}
\item 10\% CO$_2$ in Argon - $<5$~psi
\item 10\% C$_4$H$_{10}$ in Argon - $15$~psi (not used for RG-L)
\item 10\% CF$_4$ 10\% C$_4$H$_{10}$ in Argon - $15$~psi (not used for RG-L)
\item N$_2$ - $35$~psi
\item CO$_2$ - $15$~psi
\item C$_4$F$_{10}$ - $4 - 8$~psi
\end{itemize}

\subsection{Mitigations}

The 1500~gallon liquid-argon dewar and 160~liter liquid CO$_2$ dewar are used for gas supply 
only. The dewars have relief valves preventing overpressure.

Liquid nitrogen is used in both gas and liquid states. The 1500~gallon LN$_2$ dewar has a 
relief valve preventing overpressure.  
N$_2$ gas is used as a purge gas for detectors and other equipment. The purge flow is 
controlled by mass flow controllers or flow rotameters and discharges to the atmosphere.
Any work on this system must be covered by ePAS Permit to Work(s) (PTW).

\noindent
Detectors:
\begin{enumerate}
\item The Drift Chambers have both active and passive pressure protection. Pressure relief 
bubblers attached to the detector exhaust manifolds passively prevent the pressure from 
exceeding $0.125$~inch wc pressure or vacuum. The active pressure protection system consists 
of a pressure transducer, process controller, and solenoid valves that isolate the detectors 
from the gas system if the pressure goes outside the $0.025-0.125$~inch wc band. There are 
EPICS-based alarms to alert personnel of high or low pressure and flows.

The N$_2$ purge for the endplate electronics is controlled by rotometers in the 96B Gas Shed. 
This purge discharges at atmospheric pressure.

\item The ALERT AHDC drift chambers is designed to operate at a constant internal pressure. The
gas system was designed with both active and passive pressure protection. Pressure relief 
bubblers attached to the detector exhaust manifolds passively prevent the pressure from 
exceeding $0.125$~inch wc pressure or vacuum. The active pressure protection system consists 
of a pressure transducer, process controller, and solenoid valves that isolate the detectors 
from the gas system if the pressure goes outside the $0.025-0.125$~inch wc band. There are 
EPICS-based alarms to alert personnel of high or low pressure and flows.

\item The LTCC has both active and passive pressure protection. Pressure relief bubblers 
attached to the detector exhaust passively prevent the pressure from exceeding $4.00$~inch wc 
or vacuum. The active pressure protection system consists of a pressure transducer, process 
controller, and solenoid valves that isolate the detector from the gas system if pressure goes 
outside the $1.00 - 3.00$~inch wc pressure band. There are EPICS-based alarms to alert 
personnel of high or low pressures.
	
Liquid N$_2$ is used to cool the C$_4$F$_{10}$ distillation unit in the 96B Gas Shed in order 
to condense and recover C$_4$F$_{10}$ for reuse. The distillation unit has a relief valve 
preventing overpressure. The N$_2$ discharge flows through heat exchangers and vents to 
atmosphere at ambient temperature and pressure.

\item The MVT gas mixing system supplies gas to the MVT and FT gas control chassis. The system
has relief valves that prevent the gas supply and mixed gas pressure from exceeding 45 psi. A
flammable gas detector is used at the valve panel in the 96B Gas Shed to warn of gas leakage.
There are EPICS-based alarms to alert personnel of high or low pressure.
  
\item The FT calorimeter has a N$_2$ purge to prevent condensation that discharges to the 
atmosphere at ambient pressure. There are EPICS-based alarms to alert personnel of high or low 
flow.

\item The HTCC CO$_2$ purge flow is controlled by a MFC. Pressure relief bubblers attached to 
the detector volume, passively prevent pressure from exceeding $0.125$~inch wc or vacuum. There 
are EPICS-based alarms to alert personnel of high or low pressure and flows.

\item The RICH air cooling supply has pressure relief valves at the compressor output, at the 
air receiver, and at the rotameter input to prevent an overpressure condition. The N$_2$ purge 
for the aerogel discharges to the atmosphere. There are EPICS-based alarms to alert personnel 
of high or low pressure.

\end{enumerate}

\subsection{Responsible Personnel}

Individuals responsible for the gas systems are (see Table~\ref{tb:gas}):

\begin{table}[!htb]
\centering
\begin{tabular}{|c|c|c|c|c|} \hline
Name & Dept. & Phone &email & Comments \\ \hline
Engineering on call & Hall~B & 757-748-5048 & $-$ & 1st contact \\ \hline
\end{tabular}
\caption{Personnel responsible for the CLAS12 gas system.} 
\label{tb:gas}
\end{table}

