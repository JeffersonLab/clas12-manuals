\section{DAQ and Trigger}

The DAQ and Trigger systems consists of multiple VXS, VME, and other crates housing 
various readout modules such as FADC250 ADCs, TDC1190 and TDC1290 TDCs, 16-channel 
discriminators, trigger modules, and various other units. These crates are powered 
by industrial power supplies, most of them produced by Wiener, Germany.

The computer cluster contains about 30 computers located mostly in the Hall~B Counting 
House, but some computers are installed in the hall. The network consists of about 20 
switches and routers located in both the Counting House and in the hall. Backup power 
is provided by three large UPS devices, one in the Counting House and two in the hall.

Signal and power transmission is handled by a large number of copper cables interconnecting  
the various electronics modules and detector elements. A smaller number of optical cables 
are employed to transmit synchronization, time-keeping signals, and various other 
communication services throughout the experimental hall.

\subsection{Hazards} 

Hazards to personnel include the electric power supplied to the electronic components. There 
is also a fire hazard associated with cabling throughout the experimental hall.

\subsection{Mitigations}

All of the crates and chassis are commercially available and are powered from 208~V AC. 
These meet stringent safety requirements set by various qualified agencies such UL and TUV. 
Internal fans help manage thermal loads and several internal controls are implemented to 
provide limits on over-current and over-temperature excursions. All structures are grounded. 
Additionally, aluminum blank panels have been installed to limit access to the backplane on 
the rear of the chassis and on the front side where slots are unused. All power distribution 
is power-limited for current and voltage and interlocked via the Slow Controls system. All 
cables are NEC UL rated CL2 or better and conform to the 2011 edition of the NEC NFPA70 code 
requirements for fire prevention and thus, limit flame propagation in case of fire. Additionally, 
all cables are shielded and referenced to ground for added personnel and equipment safety.

Any work on this system must be covered by ePAS Permit to work(s) (PTW).

There are possible electrical hazards if a malfunctioning electronics component is replaced. 
The associated task hazard analysis concluded that the consequence level is low, the 
probability level is low, and the risk code is 1. The mitigation for these electrical jobs is 
to place the equipment in Mode 0 (de-energized) when replacing or repairing hardware during 
routine maintenance.

\subsection{Responsible Personnel}

Individuals responsible for the DAQ and Trigger system are:

\begin{table}[!htb]
\centering
\begin{tabular}{|c|c|c|c|c|} \hline
Name&Dept.&Phone&email&Comments \\ \hline
Sergey Boyarinov&Hall~B&757-232-6221&\href{mailto:boiarino@jlab.org}{\nolinkurl{boiarino@jlab.org}}&1st contact \\ \hline
 \end{tabular}
\caption{Personnel responsible for the CLAS12 DAQ and Trigger system.} 
\label{tb:daq}
\end{table}

