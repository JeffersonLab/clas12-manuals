\section{PRad Target System}

The Proton Charge Radius experiment in Hall B (PRad) utilizes a windowless, hydrogen gas jet target constructed by the Jefferson Lab Target Group. Room temperature
hydrogen flows through a 25 K heat exchanger attached to a mechanical cryocooler, and
accumulates in a 50 mm diameter, 40 mm long copper target cell located within a small
(< 1 m3) differentially pumped vacuum chamber. The target cell, which is suspended
from the top of the vacuum chamber using a precision, 5-axis motion mechanism, has
25 µm-thick Kapton covers at both ends with 4 mm orifices for the electron beam. The
covers are easily detachable, so different orifice sizes can be used to examine the effects of
possible beam halo. The gas is pumped from the chamber using two large turbomolecular
vacuum pumps with a combined pumping speed of 5700 l/s. The gas pressure within
the cell is measured by a precision capacitance manometer and is expected to be approximately 0.6 torr 
during the PRad experiment, giving in an areal density of about $10^{18}$ H$_2$/cm$^2$. 
Two additional turbo pumps are attached to the upstream and downstream 
ends of the vacuum chamber to maintain a beamline vacuum less than 10$^{-5}$ torr.
Hydrogen gas is metered into the target system using a precision, room-temperature
mass flow controller, while gas pumped from the chamber is exhausted outside the experimental hall via the Hall B vent header. Mechanical interlocks are used to stop the flow
of hydrogen gas in the event any of vacuum, pressure, or temperature failures. These
interlocks ensure that the quantity of H$_2$ in the chamber is always less than 30 mg

\subsection{Hazards}
The target utilizes hydrogen gas as the target material, while the cryocooler uses helium
gas as a coolant. Therefore, a potential ODH risk is present. The total inventory of the
H2 gas in the target system is about 1 liter, while the volume of helium gas necessary
to operate the PTR is 81 liters. The volume of Hall B is approximately $1.2 \times  10^7$
liters (437,500 ft$^3$). Release of the targets hydrogen/helium gases in this area would be
completely negligible, decreasing the Hall B oxygen levels by less than 0.001\%. Therefore,
the gas jet target does not impact the ODH classification of this location. High-pressure
cylinders of hydrogen, each containing approximately 8000 standard liters, will be used
to supply gas to the target.
A few control electronics will be attached to an uninterruptable power supply (UPS),
including the LS336 temperature controller and scattering chamber vacuum readout.
This is primarily for monitoring reasons, because the system is designed to be intrinsically safe in the event of power outages. In all cases, the cryocooler simply turns off
and the target slowly warms to room temperature. The (Normally Closed) valves that
provides hydrogen gas to the target close, the upstream and downstream gate valves on
the chamber close, and the turbo pumps spin down.
The volume of the vacuum chamber is approximately 0.26 m$^3$, representing a stored
energy of 26 kJ. This is less than the 100 kJ limit imposed by the Jefferson Lab EH\&S
manual for buckling analysis. It is also exempt from Code welding/brazing requirements.
There are no thin windows on the chamber.
Two new components in the target system fall under the purview of pressure systems
safety:
1) The gas handling panel in the experimental hall,
2) Chilled water lines between a pair of water chillers and the vacuum pumps and the
cryocooler compressor.
The gas panel is constructed in accordance with ASME B31.3 (2012). It is supplied
with gas from a standard H2 cylinder, with regulator, located in the Hall B gas pad
using existing pipes between the pad and the experimental hall. Relief valves located at
the exit of the regulator ensure a maximum gas pressure of 30 psig in the gas handling
system
Chilled water is supplied to the turbo pumps and cryocooler compressor from a
pair of NESLAB ThermoFlex 10000 chillers, each with a maximum outlet pressure of
60 psi. All fittings, lines and other components in the chilled water system are rated
to a minimum working pressure of at least 90 psig. The chilled water system is now
considered a low hazard system (stored energy less than 1000 ft-lbf) and is exempted from
the pressure system program requirements now in effect. Nonetheless, it was constructed
to and is in compliance with ASME B31.3 (2012) Cat D.
The PRad target utilizes hydrogen gas, which is flammable in air over a range
of concentrations from 4\% to 75\%, by volume. The quantity of hydrogen gas inside
the PRad target system (comprising the gas panel, internal piping and target cell, and
target chamber) is about 1 standard liter, or 0.09 grams. Therefore the system may be
classified as a Class 0 risk (Q<0.6kg). All potential ignition sources on the gas panel
(pressure transducers and flow controllers) meet CLASS I DIV 1 GROUPS A, B, C,
\& D standards. All thermometers inside the chamber operate at very low voltage and
currents. A 100 W heater is used to control the cryocooler temperature at about 25 K.
It is automatically de-energized by the 1 torr pressure switch on the scattering chamber.
Therefore the maximum quantity of H2 that can be in contact with this potential ignition
source is only 26 mg. The chamber will be evacuated of hydrogen and purged with an
inert gas such as nitrogen or argon before it opened to air.
A standard cylinder of hydrogen (approx. 8200 standard liters) is used to provide
a constant supply of H2 gas to the target system. This cylinder is part of the standard
Hall B liquid hydrogen target, is located in the Hall B gas pad, away from any ignition
sources. It is capped when not in use and properly labeled Danger-Flammable Gas. The
lines leading from the cylinder to the target installation have also been used for several
years for the Hall B liquid hydrogen target. Any necessary extensions to these lines will
be constructed in accordance to ASME 31.3 (2012).
All pump exhausts and relief lines from the target are attached to the same Hall B
vent header that has been utilized for the Hall B liquid hydrogen target. A steady purge
of inert nitrogen gas is used to prevent a flammable mixture in the vent. Any necessary
piping between the PRad target installation and the vent header will be constructed
in accordance to ASME 31.3 (2012). The vent header will be properly labeled Danger
Flammable Gas. The Hall B flammable gas monitoring system will be in operation
throughout the PRad experiment.
No cold portions of the PRad target are accessible by personnel. Due to the windowless nature of the target, no condensed cryogenic fluids can be accumulated within
its volume, and the total quantity of vapor is only about 0.03 g.
Cernox thermometers monitor the temperature of the fluid at numerous locations,
including the cryocooler cold head, the copper target cell, and the vapor inside the cell
itself. The temperature of the cold head is regulated at about 25 K using a Lake Shore
Model 336 temperature controller. Normally open contacts on the controller turn off the
cryocooler before the temperature reaches the condensation point of H2, about 22.3 K at
1 atm.
Frozen contaminates in the H2 gas could impede or stop the flow of gas to the target
cell. However this does not represent a hazard, as no condensed fluids exist in the system.
Nevertheless precautions are made to ensure the purity of the target gas for the PRad
experiment. Detailed gas handling procedures will be utilized to ensure that no gases
other than hydrogen (or helium, for purges) are present in the system. Only high purity
hydrogen gas (research or scientific grade) will be used in the target, and this will be
introduced into the cell through a purifier installed on the gas panel for further removal
of water and oil.


\subsection{Mitigations}
The high pressure hydrogen cylinders will be located in the Hall B gas shed and connected
to the target gas panel using existing lines in Hall B. The target is designed such that
upon power failure the Cryocooler shuts off, gas valves close. The vacuum system of the
target is designed to minimize volume and avoid any thin windows. A pressure switch
on the chamber will automatically shut off hydrogen gas to the chamber at 1 torr. A
check valve with Cv=0.66 is installed on the chamber to prevent overpressure, should
the switch fail. All pressure systems of the target have been design and constructed in
compliance with relevant ASME pressure system codes. Hazards will be mitigated by
routine inspection, testing, and replacement of system components. The flammable gas
hazard will be mitigated by flammable gas and hydrogen monitoring, use of non-sparking
tools, minimization of ignition sources, compliance with ASME 31.12 for hydrogen piping,
proper posting of Flammable Gas Area, inerting system prior to maintenance and repair,
leak testing system before operating system, and following approved procedures during
operation. Cryogenic system hazards will be mitigated by temperature and pressure
alarms, temperature interlocks, minimization of target gas and volume, and Documented
gas handling procedures.

%The emergency and interlock response procedures are as follows:
%• Flammabale gas alarm If the flammable gas alarm goes off, notify others, press the
%KILL switch on the target control panel, and evacuate the area.
%• ODH alarm If the ODH alarm sounds, notify others, press the KILL switch on the
%target control panel, and evacuate the area.
%• Low temeperature interlock The Lake Shore 336 temperature controller has normally open relays that will turn off the cryocooler if the temperature of the target
%gets too close to the condensation point of hydrogen. The system will slowly warm.
%This is not considered an emergency. Contact the Target Expert.
%• Vacuum interlock A 1 torr vacuum switch will close electric valves HPA-1 and HPA2 located on the target gas panel. This is not considered an emergency. Contact
%the Target Expert.
%• Vent purge interlock A pressure switch is installed on the hydrogen vent line to
%ensure that it is adequately purged with an inert gas such as nitrogen. An alarm
%in the counting house will sound if this switch energizes. Check that the purge line
%is properly connected and that gas is flowing into the vent. If it is not, contact the
%Hall B work coordinator. If gas is flowing but the switch is still energized, contact
%the Target Expert.
%The inspection and maintenance schedule is as follows:
%• Every run
%Visually inspect all process piping.
%Leak check all process piping.
%Confirm operation of electronic monitoring and control hardware.
%Confirm all thermometer readouts ( 295 K at room temperature).
%Confirm operation of pressure switches PAH1.
%Confirm operation of check valve CV1.
%Zero pressure transducers PI1 and PI2 and calibrate against their corresponding
%pressure gauges.
%Inspect filter/purifier element FP1.
%Review and if necessary update the safety and operation manuals for the PRad
%target.
%Visually inspect vacuum pumps.
%Visually inspect the PTR, its compressor and Aeroquip connections.
%Visually inspect water-cooling piping for PTR compressor and turbo pumps.
%Confirm operation of flow switch for PTR compressor.
%Confirm PTR helium charge.
%• Every year
%Have pressure gauges PI1 and PI2 calibrated by a vendor certified to JLab standards.
%• Other
%Inspect/test all relief and check valves in the system every two years.
%Replace adsorber in PRT compressor after 20,000 hours of operation (consult PRT
%manual).

\subsection{Responsible personnel}
The target system will be maintained by the Hall B engineering group.





