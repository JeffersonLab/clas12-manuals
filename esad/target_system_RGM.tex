\section{Target System}

The target system used for RGM is the Saclay cryo target.  This target system has been used in Hall B throughout the 6 GeV and 12 GeV era.

RGM will use the following targets inside the Saclay cryo target system:
\begin{itemize}
\item	Liquid:
   \begin{itemize}
	\item	Hydrogen
	\item		Deuterium
	\item		Helium 4
	\item		Argon
    \end{itemize}
\item		Solid
   \begin{itemize}
	\item		Tin
	\item		Carbon
	\item		Calcium
   \end{itemize}
\end{itemize}
	
The targets will be set up in the following configurations:
\begin{itemize}
\item		5 cm long liquid cell for Hydrogen, Deuterium, and Helium 4
\item		5 mm long liquid cell for argon with carbon and tin foils that can be moved in and out of the beam
\item		Solid calcium disk
\end{itemize}

The targets are housed in a vacuum vessel along with the cryogenic system.  A scattering chamber is installed around the target cell area.  This is made from Rohacell foam with a wall thickness of 6.5 mm.  Aluminum windows are used at the entrance and exit of the liquid cells, and at the exit of the scattering chamber.

The details of all components, such as windows and cells, are shown on the beam line drawing, including thicknesses and locations.  The beam line drawings can be found at:

https://clasweb.jlab.org/wiki/index.php/User:Cwiggins
               
\subsection{Hazards} 

The cryogenic target contains a condensed cryogenic fluid and is considered a pressure vessel. 
Sudden warming of the target due to a vacuum breach could result in rapid expansion of the 
target fluid. The system is designed to safely vent the excess pressure. Failure of the 
foam scattering chamber, or the thin window of the scattering chamber could produce a loud 
noise and could result in a failure of the target integrity. The target utilizes flammable 
gas (hydrogen) during operation. Failure of the system could release flammable gas into the 
hall. The target gases and the helium used in the target refrigerator are potential 
ODH risks, and failure of either system could reduce the oxygen levels in the hall.

The solid target assembly contains helium gas fed from a tank with a 18 psi relief valve.  The target operates in a vacuum chamber, so the total pressure difference possible across the cell is 18 psi + 14.7 psi = 32.7 psi.  The cell is considered a pressure vessel. If the Kapton cell ruptures, the helium would vent into the target vacuum space and the vacuum pumps would turn off.  If the vacuum space pressure increases to 7 psi, the helium will go out of the vacuum space relief valve and be discharged out of the Hall.  No helium would enter the Hall. 


\subsection{Mitigations}

The cryotarget system has been used for about 15~years with the CLAS detector in Hall-B. The 
design and construction of the new target cell and the scattering chamber are in accordance 
to AMSE standards. During operation, the foam scattering chamber, and the thin 
window are surrounded by the Hall-B CLAS12 Central Detectors and are therefore difficult to 
access. A protective shield will be placed around the scattering chamber whenever the target 
is retracted from the Central Detector system and is under vacuum. Personnel working near the 
target shall wear hearing and eye protection whenever the foam extension and window are 
exposed and the system is under vacuum. No cold cryogenic components are accessible by personnel. 

Two flammable gas detectors are installed near the target location, one above the right hand 
side of the gas system rack and another above the cryostat. In case of a detected leak, the 
control system will immediately warm up the whole target system and empty the target cell.

Relief valves installed in parallel with all of the remote control valves ensure that the 
safety system is entirely passive. In case of an increased pressure in the vacuum chamber, 
the cryotarget controller will stop all vacuum pumps on the cryotarget, close the 
target supply valves, and sound an alarm (located on top of the terminal 
in the Hall-B Counting House).

The quantity of flammable gas (H2) is less than 80~g and is therefore considered a class-O 
installation (<600~g) and the rules and regulations for this installation shall be followed, 
notably:

\begin{itemize}

\item The area shall be posted ``Danger Flammable Gases.  No Ignition Sources".

\item Combustibles and ignition sources shall be minimized within 10~ft or 3~m of target's gas 
handling equipment and piping.
\end{itemize}

The target does not operate in a confined space, and the total quantity of hydrogen/helium in 
the system is under 1000 standard liters. This presents a negligible oxygen deficiency risk in 
Hall~B and therefore is a class-0 ODH installation.

Hydrogen shall be loaded into the system by qualified personnel only, and those personnel shall 
follow approved operational gas handling procedures.   Upon loss of target gas pressure, the 
control system shall automatically warm up the target to keep the pressure above atmospheric to 
prevent contamination. 

The target control software includes numerous alarms (temperature, pressure, vacuum, heater 
power, etc.) to alert users to potential problems. 


\subsection{Responsible Personnel}

The target system will be maintained by the Hall~B Engineering Group.  

\begin{table}[!htb]
\centering
\begin{tabular}{|c|c|c|c|c|}
\hline
 Name&Dept.&Phone&email&Comments \\ \hline
R. Miller &Hall~B&(757) 822-9586&\href{mailto:rmiller@jlab.org}{\nolinkurl{rmiller@jlab.org}} &1st contact \\ \hline
D. Insley & Hall~B&x7566&\href{mailto:dinsley@jlab.org}{\nolinkurl{dinsley@jlab.org}}  &3rd contact \\ \hline
K. Bruhwel& Hall~B&x7868&\href{mailto:bruhwel@jlab.org}{\nolinkurl{bruhwel@jlab.org}}&2nd contact \\ \hline
Engineering on call & Hall~B&(757)-748-5048&& 4th contact  \\ \hline
\end{tabular}
\caption{Personnel responsible for the CLAS12 target system.} 
\label{tb:target}
\end{table}
