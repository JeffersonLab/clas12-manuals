\section{Target System}

The ALERT experiment will use a variety of gas targets (hydrogen, deuterium, and helium) at
room temperature at up to 4~atm pressure (4~atm = 58.78 psig). The target cell is a
45-cm-long aluminum-coated Kapton straw of 0.6 cm diameter and 50 $\mu$m wall thickness. The
straw has entrance and exit windows of 30-$\mu$m thick aluminum. The target will be mounted
inside the center bore of the ALERT detector along the beamline. There will be a 1 cm air gap
between the upstream beamline and the ALERT target. The upstream beam pipe includes a 30~$\mu$m
thick aluminum window. ALERT will have a helium extension tube on the downstream end with a
30 $\mu$m aluminum window. There will be a 4 cm air gap between this window and the 75 $\mu$m
thick window on the downstream beam pipe. The details of all components, such as windows and cells,
are shown on the beamline drawing, including thicknesses and locations. These drawings can be found
at Ref.~\cite{engineering-page}.

\subsection{Hazards} 

The main hazard is a potential rupture of the target cell or the target entrance and exit windows.
If the target is filled with $^4$He at the time, the only hazard would be a loud noise and potential
damage to the ALERT detector surrounding the target due to a pressure shock. If the target is filled
with $^1$H or $^2$H, an additional hazard will be the presence of these flammable gases within the
ALERT environment. The total gas volumes involved are too small to create a ODH risk.

\subsection{Mitigations}

The design and construction of the gas target is in accordance to ASME B31.3 process piping code, except
downstream of the isolation valve. This section includes the target cell and is qualified by equivalent
measures. These include limiting the amount of flammable gas in this section and preventing excess flow
into this section from the buffer volume in case of a sudden pressure loss due to a rapture of the target
cell or its windows. A flame arrestor prevents the ignition of flammable gas upstream of this section
(upstream of the isolation valve).

During operation the target straw and the thin entrance and exit windows are surrounded by the ALERT
detector and are therefore difficult to access. The target will only be filled with gas in excess
of 1~atm pressure when installed inside the CLAS12 solenoid.

Two H$_2$/D$_2$ gas detectors are installed near the target location, one above the right hand side of the
electronics rack and another above the cryostat. In case of a detected leak, the control system will
immediately shut off the supply of gas to the target.

A pressure sensor is mounted downstream of the isolation valve to detect a sudden loss of pressure and
immediately shut off the gas supply. The quantity of flammable gas (H$_2$ or D$_2$) downstream of the
isolation valve is less than 1000~ft lbs in energy (1355~J). The ratio of energy per volume for hydrogen
or deuterium is 14.5~J/cm. This includes chemical heat of combustion (13~J/cm$^3$) and mechanical explosion
energy (1.5~J/cm$^3$, Brode equation). For the limit of 1355~J this corresponds to 13.3~cm$^3$ of deuterium
(or hydrogen) gas at 7~atm (93.4~cm$^3$ at 1 atm).

This volume limit in the section qualified by equivalent measures will be achieved by limiting the volume
of the target cell plus the capillary supply tubing up to the isolation valve to 13.3~cm$^3$. To limit the
amount of gas flow into the target cell, the supply tubing inner diameter will be capillary and an excess
flow valve will limit the flow in case of sudden loss of pressure in the target cell due to bursting.

\begin{itemize}
\item The area shall be posted ``Danger Flammable Gases. No Ignition Sources'',
\item Combustibles and ignition sources shall be minimized within 10~ft or 3~m of target’s gas handling
  equipment and piping.
\end{itemize}

The target does not operate in a confined space, and the total quantity of hydrogen/helium in the system is
under 1000 standard liters. This presents a negligible oxygen deficiency risk in Hall~B and therefore is a
class-0 ODH installation.

Hydrogen shall be loaded into the system by qualified personnel only, and those personnel shall follow
approved operational gas handling procedures. Upon loss of target gas pressure, the control system shall
automatically shut off the gas supply.

The target control software includes numerous alarms (temperature, pressure, vacuum, etc.) to alert users
to potential problems.

\subsection{Responsible Personnel}

The target system will be maintained by the Hall~B Engineering Group (see Table~\ref{tb:target}).  

\begin{table}[!htb]
\centering
\begin{tabular}{|c|c|c|c|c|}
\hline
Name & Department & Phone & email & Comments \\ \hline
Engineering on call & Hall B & 757-748-5048 & & 1st contact \\ \hline
D. Insley & Hall~B  &757-897-9060&\href{mailto:dinsley@jlab.org}{\nolinkurl{dinsley@jlab.org}} & 2nd contact \\ \hline
R. Miller & Hall~B  & x7867      &\href{mailto:rmiller@jlab.org}{\nolinkurl{rmiller@jlab.org}} & 3rd contact \\ \hline
\end{tabular}
\caption{Personnel responsible for the CLAS12 ALERT target system.} 
\label{tb:target}
\end{table}
