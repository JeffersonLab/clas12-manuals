\section{Electromagnetic calorimeter}

The Electromagnetic Calorimeter (HYCAL)will be located approximately 7.4m (run
configuration) or 8.7m (calibration configuration) upstream of the center of the CLAS12.
It consists of 1700 lead glass and lead tungstate detector modules, each with photomultiplier tubes with readout enclosed inside a temperature-controlled enclosure. Each
module has a PMT supplied with high voltage. In addition, an LED-based light monitoring system is used to deliver a pulse of
light to each module via a fiber optic cable. The HYCAL will sit in two positions along
the beamline. In the run configuration, HYCAL will sit on a stationary cart. In the
calibration configuration, it will be mounted on a transporter, enabling motion in
the horizontal and vertical directions. The detector has overall dimensions of 1.5m 1.5m
and will be centered on the beamline during production data taking. A 4cm 4cm hole
at the center of the detector will allow the passage of the primary electron beam to the
beam dump.  
\subsection{Hazards}
Hazards associated with this device are electrical shock or damage to the PMTs if the enclosure is opened with the HV on. 
There is also a hazard associated with the coolant leak inside the enclosure.
\subsection{Mitigations}
Whenever any work has to be done on the calorimeter, whether it will be opened
or not, HV and LV must be turned off. They are interlocked with the access doors of the calorimeter.
Turn the chiller off if the enclosure will be opened for maintenance. The chiller is interlocked with the internal leak detector.
Any large (more than a couple of degrees in C) temperature changes must be investigated to make
sure that there are no leaks.
\subsection{Responsible personnel}

to be added

