\section{Forward Time-of-Flight System}

The Forward Time-of-Flight System (FTOF) is mounted on the Forward Carriage in Hall~B. 
In each of the six sectors of CLAS12, the FTOF system is comprised of three arrays of 
counters, named panel-1a, panel-1b, and panel-2. Each of the panels consists of a set 
of rectangular scintillation counters with a PMT on each end. The panel-1a and panel-1b 
arrays are located at forward angles (roughly $5^\circ$ to $35^\circ$) and the panel-2 
arrays are located at larger angles (roughly $35^\circ$ to $45^\circ$). In each sector 
the panel-1a arrays contain 23 counters, the panel-1b arrays contain 62 counters, and 
the panel-2 arrays contain 5 counters.

\subsection{Hazards} 

There are two hazards associated with the FTOF system related to i) the high voltage (HV)
system used to energize the counter PMTs and ii) access to the counters during testing 
operations.

The HV power supplies for each FTOF sector are either CAEN 1527LC mainframes or CAEN 4527 
mainframes outfitted with negative polarity 24-channel A1535N modules. The typical settings
for each channel are: $V=-2000$~V, $I=350$~$\mu$A. These supplies are located on the north 
and south sides of each level of the Forward Carriage behind each sector of counters. There 
are two hazards associated with the HV system when energized that must be mitigated. The 
first is the electrical hazard and the second is the potential damage to PMTs if a light 
leak is introduced in the counter wrapping material when the PMT is energized.

The panel-1b and panel-1a counters are positioned between the PCAL and LTCC detectors on 
the Forward Carriage. Therefore they are not accessible for hands-on testing. However, 
the panel-2 counters are accessible for hands-on testing when the Forward Carriage is 
pulled back into its maintenance position. The panel-2 counters in the S1, S2, S3, and S4 
positions can then be accessed by manlift and the panel-2 counters in the S5 and S6 
positions can be accessed by either manlift or ladders. When testing the panel-2 counters 
in such an operation there are fall hazards that must be mitigated.

\subsection{Mitigations}

The electrical hazard associated with the HV system would be to receive an electrical 
shock. However, the design of the HV system for the FTOF is such that the chance to 
receive an electrical shock is minimal. The electrical hazards are mitigated by the use 
of properly rated RG-59 cables that are terminated at the voltage divider end and the HV 
supply end. As well, the HV supplies are grounded to their electronics racks. The bigger 
issue would be damage to a PMT if improper contact with the counter surface were to occur 
that introduced a sizable light leak in the counter wrapping. However, the hazards in such 
a situation are minimal in that the HV system is designed to shutdown any channels that 
show an over-current condition, thereby protecting the system hardware. 

Only authorized FTOF system personnel are allowed to work on the counters during hands-on
testing when the Hall~B configuration allows such work. For these individuals using ladders 
or manlifts, they are required to have all appropriate training including manlift and
harness training, ladder training, and fall protection training. All work is carried out 
in conjunction with input from the FTOF Group Leader and the Hall~B Work Coordinator.

\subsection{Responsible Personnel}

Individuals responsible for the FTOF system are:

\begin{table}[!htb]
\centering
\begin{tabular}{|c|c|c|c|c|} \hline
Name              & Dept.  & Phone          & email & Comments \\ \hline
FTOF/CTOF on call & Hall B & (757)-344-7204 &       & 1st contact \\ \hline
D.S. Carman       & Hall B & x5586          & \href{mailto:carman@jlab.org}{\nolinkurl{carman@jlab.org}} & 2nd contact \\ \hline
\end{tabular}
\caption{Personnel responsible for the CLAS12 FTOF system.} 
\label{tb:ftof}
\end{table}

