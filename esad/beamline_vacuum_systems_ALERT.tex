\section{Beamline}

The control and measurement equipment along the Hall~B beamline consists of various 
elements necessary to transport the beam with the required specifications onto the 
production target and the beam dump, and simultaneously to measure the properties of 
the beam relevant to the successful implementation of the physics program in Hall~B. 

The beamline in the hall provides the interface between the CEBAF accelerator and the 
experimental hall. All work on the beamline must be coordinated with both the Physics 
Division and the Accelerator Division in order to ensure safe and reliable transport of 
the electron beam to the dump. The Accelerator Division has the primary responsibility 
of delivering the electron beam to the experimental target and designated dumps.

\subsection{Hazards} 

Along the beamline various hazards can be found. These include radiation areas, vacuum 
windows, high voltage, and magnetic fields.

\subsection{Mitigations}

All magnets (dipoles, quadrupoles, sextupoles, beam correctors) and beam diagnostic devices 
(BPMs, scanners, beam loss monitors, viewers) necessary to transport and monitor the beam 
are controlled by the Machine Control Center (MCC) and/or Hall~B personnel through EPICS 
\cite{epics}. The detailed operational procedures for the Hall~B beamline are essentially 
the same as those for the CEBAF machine and beamline.

Personnel who need to work near or around the beamline should keep in mind the potential 
hazards:
\begin{itemize}
\item Radiation ``Hot Spots" - marked by an ARM or RadCon personnel,
\item Vacuum in beamline elements and other vessels,
\item Thin-windowed vacuum enclosures ({\it e.g.} the scattering chamber),
\item Electric power hazards in the vicinity of magnets, and 
\item Conventional hazards (fall hazard, crane hazard, etc.). 
\end{itemize} 
Any work on this system must be covered by ePAS Permit to Work(s) (PTW).

These hazards are noted by signs and the most hazardous areas along the beamline are 
roped off to restrict access when operational ({\it e.g.} around the magnets). Signs are posted 
by RadCon for any hot spots. Surveys of the beamline and surrounding areas will be 
performed before any work is done in these areas. The connection of leads to magnets have 
plastic covers for electrical safety. Any work around the magnets will require 
de-energizing the magnets. Energized magnets are noted by red flashing beacons. Any work 
on the magnets requires the ``Lock and Tag" procedures~\cite{esh} by appropriately trained
and certified electrical workers. Additional safety information can be obtained from the 
ES\&H Manual~\cite{esh}.

\subsection{Responsible Personnel}

The beamline requires both Accelerator and Physics Division personnel to maintain and 
operate (see Table~\ref{tb:beam}). It is very important that both groups stay in contact
with each other to coordinate any work on the Hall~B beamline. 

\begin{table}[!ht]
\centering
\begin{tabular}{|c|c|c|c|c|} \hline
Name&Dept.&Phone&email&Comments \\ \hline
Expert on call& &757-303-3996&& 1st contact \\ \hline
E. Pasyuk     & Hall~B&x6020&\href{mailto:pasyuk@jlab.org}{\nolinkurl{pasyuk@jlab.org}}&2nd contact \\ \hline
M. Tiefenback & Accel.&x7430&\href{mailto:tiefen@jlab.org}{\nolinkurl{tiefen@jlab.org}}& beamline optics\\ \hline
K. Price      & Accel.&x7067&\href{mailto:kprice@jlab.org}{\nolinkurl{kprice@jlab.org}}&Contact to OPS \\ \hline
\end{tabular}
\caption{Responsible personnel for the Hall B beamline.} 
\label{tb:beam}
\end{table}

\section{Vacuum System}

The Hall~B vacuum system for the RG-L (ALERT) experiment consists of two segments: the beam
transport line to the experimental target, consisting of 1.5~in to 2.5~in diameter
beampipes and the vacuum beamline to the Hall~B electron dump consisting of 
2~in to 6~in diameter beampipes. The vacuum spaces are physically isolated from each
other and the vacuum level can be monitored independently using cold cathode gauges. The
vacuum in the system is provided by a set of roughing, turbo, and ion pumps, and is
maintained at a level of better than $10^{-5}$~Torr. 

\subsection{Hazards} 

Hazards associated with the vacuum system are due to rapid decompression in case of a window 
failure. Loud noises can cause hearing loss. 

\subsection{Mitigations}

All personnel working in the vicinity of the entrance and exit windows are required to wear 
hearing protection. Warning signs must be posted in that area. In addition, all vacuum vessels 
and piping are designed as pressure vessels.  Any work on this system must be covered by ePAS
Permit to Work(s) (PTW).

\subsection{Responsible Personnel}

The vacuum system will be maintained by the Hall~B Engineering Group (see
Table~\ref{tb:vacuum}).  

\begin{table}[!htb]
\centering
\begin{tabular}{|c|c|c|c|c|} \hline
Name&Dept.&Phone&email&Comments \\ \hline
Engineering on call & Hall~B& 757-748-5048&& 1st contact  \\ \hline
D. Insley & Hall~B  &757-897-9060&\href{mailto:dinsley@jlab.org}{\nolinkurl{dinsley@jlab.org}}  &2nd contact \\ \hline
 \end{tabular}
\caption{Personnel responsible for the Hall~B vacuum system.} 
\label{tb:vacuum}
\end{table}

