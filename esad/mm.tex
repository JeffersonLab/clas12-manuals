\section{Micromegas Vertex Tracker}

The CLAS12 Micromegas Vertex Tracker apparatus is comprised of the Barrel Micromegas 
Tracker (BMT) and the Forward Micromegas Tracker (FMT), which despite their different 
shapes (cylinders or disks), are the same type of detector and therefore have the same 
list of hazards. Both subsystems are gaseous detectors, the only difference with respect 
to the hazards is the use of different gases for the BMT and the FMT.

The BMT is composed of 3 double-layers of resistive cylindrical Micromegas detectors, each 
layer is divided into 3 sectors, for a total of 18 detectors. In combination with the 
Silicon Vertex Tracker, the BMT covers the polar angle region from 35$^\circ$ to 125$^\circ$ 
around the target. Micromegas detectors are double-stage gaseous detectors. The gas  
used for the BMT is a mixture of 90\% argon and 10\% isobutane. Even though isobutane 
is a flammable gas, the amount of gas in use at any given time is well within the range of a 
class-0 gaseous device.

The BMT is powered by two high voltages up to 2000~V, although the current limit for both 
HV is extremely low. The detectors are read out through 1.5-m-long flex cables by Front-End 
Units (FEU). These electronic cards contain the customized DREAM ASICs in order to sample the 
detector signal and a flash-ADC to digitize it and send it to the network. The FEUs are placed 
inside customized crates, on the back of the support tube holding the MVT. They are powered 
through low voltage, and kept within the 40$^\circ$C to 60$^\circ$C temperature range using a 
simple set of fans and tubing.

The FMT is composed of 6 flat resistive Micromegas detectors. Each detector is divided in two 
zones (inner and outer). The FMT covers the polar angle region from 6$^\circ$ to 29$^\circ$ 
from the target. Micromegas detectors are double-stage gaseous detectors. The gas 
used for the BMT is a mixture of 80\% Argon, 10\% CF$_4$, and 10\% isobutane. Even though 
isobutane is a flammable gas, the amount of gas in use at any given time is well within the 
range of a class-0 gaseous device.

The FMT is powered by three high voltages up to 2000~V, although the current limit for all 
three HV is extremely low ($<$1~mA). The detectors are read out through 2.2-m-long flex cables 
by Front-End Units (FEU). These electronic cards contain the customized DREAM ASICs in order to 
sample the detector signal and a flash-ADC to digitize it and send it to the network. The FEUs 
are placed inside customized crates, located on the back of the support tube holding the MVT. 
They are powered through low voltage, and kept within the 40$^\circ$C to 60$^\circ$C temperature 
range using a simple set of fans and tubing.

All hazards and mitigation options for the FMT are the same as for the BMT. Even though the 
shapes of the detectors vary, they are almost identical in principle. The gas mixture is 
however different, but the amount of flammable gas (isobutane) is almost the same.

\subsection{Hazards} 

Hazards to personnel include the use of flammable gas, high voltage, and the low voltage that 
powers the readout electronics. During the installation phase, mechanical hazards include the 
risks associated with the weight of the MVT, including its support tube, as well as the work at 
height in order to access the LV and gas control crates.

Hazards to the MVT detectors themselves include mechanical damage, gas leaks, and gas 
over-pressure. Also, there is a risk of damage to the MVT during installation in the solenoid 
bore.

Hazards concerning the MVT Front-End Units include: wrong LV settings that could damage the FEUs, 
and absence of cooling or cooling failure that would overheat the cards.

\subsection{Mitigations}
Any work on this system must be covered by ePAS Permit to work(s) (PTW).

Fire hazards (equipment and personnel):
\begin{itemize}
\item Use of flammable gas: All MVT detectors use 10\% isobutane, which is flammable. Mitigation: 
the amount of isobutane in the system is very limited. For both the BMT and FMT detectors, the 
total combustion energy is equivalent to less than 12~g of hydrogen, which makes it a class-0 
gas system (class-1 starts at 600~g).
\end{itemize}

Electrical hazards (personnel):
\begin{itemize}
\item High Voltage: high voltage up to 2500~V are used routinely for all detectors. Mitigation: 
very low current limit (10~$\mu$A) is set. All mechanical structures are properly grounded.
\item Low Voltage: In order to power up the front-end electronics, low voltage at 4.5~V 
with 60~A per crate is provided. Mitigation: voltage is low enough not to be a danger to personnel. All 
mechanical structures are properly grounded. All cables and connectors are certified for this 
rating and shielded.
\end{itemize}

Mechanical hazards (personnel):
\begin{itemize}
\item Heavy object (Total: 200~kg), handled with an overhead crane. Mitigation: job done by 
trained JLab staff according to and compliant with the Jefferson Lab EHS\&Q manual.
\item Work at height for access to gas control and LV crate (located at 2.5~m height) on the 
moving cart. Mitigation: use of certified step ladder provided by JLab.
\end{itemize}

Other hazards to MVT:
\begin{itemize}
\item Detectors: gas over-pressure, gas leaks, and mechanical damage. Mitigation: gas control 
system with over-pressure and leak limits. Protection covers (1~mm carbon shell) in order to 
avoid as much as possible damage to the detectors before installation and operation. The FMT 
detectors can be dismounted and repaired in case of accidental damage to the drift electrode. 
Since they are tightly stacked, only one such electrode is exposed and at risk, the rest of 
the stack is protected by the first detector.
\item Installation of Central Tracker: Damage to the MVT during installation in the solenoid 
bore. The CTOF is the closest detector to MVT. Mitigation: Parts of the MVT that are radially 
farthest outwards are surveyed prior to insertion into the solenoid to ensure that there is 
no interference with adjacent CTOF components. There is a large clearance ($\sim$11~mm) between 
the MVT and the adjacent detector, CTOF. The insertion will be achieved on a precision rail 
system that is used for the target insertion. There will be a constant visual check as the MVT 
is inserted into the solenoid bore. A linkage mechanism on the upstream end of the SVT/MVT allows 
for the adjustments in pitch and yaw if needed. Finally, the operation will be performed by 
trained personnel with several years of experience and familiar with similar positioning.
\item Electronics: wrong LV settings and absence of cooling or cooling failure. Mitigation: Slow 
Controls read-back of LV settings before turning on the front-end electronics. Cooling is also 
checked by the Slow Controls system, and is interlocked so that the electronics cannot be turned 
on when the cooling is off. Also, temperature sensors are present on the front-end cards and are 
directly interlocked so that if temperature goes beyond a predefined threshold, the cards are 
gracefully shut-down automatically.
\end{itemize}

\subsection{Responsible Personnel}

Individuals responsible for the MVT system are:

\begin{table}[!htb]
\centering
\begin{tabular}{|c|c|c|c|c|} \hline
Name&Dept.&Phone&email&Comments \\ \hline
Expert on call& &&& 1st contact \\ \hline
Y. Gotra &JLab&757-269-5571&\href{mailto:gotra@jlab.org}{\nolinkurl{gotra@jlab.org}}&2nd contact \\ \hline
	R. Paremuzyan&JLab&757-269-7539&\href{mailto:rafopar@jlab.org}{\nolinkurl{rafopar@jlab.org}}&3rd contact \\ \hline
M. Defurne&Saclay&+33169083237&\href{mailto:maxime.defurne@cea.fr}{\nolinkurl{maxime.defurne@cea.fr}}&4th contact \\ \hline
\end{tabular}
\caption{Personnel responsible for the CLAS12 MVT system.} 
\label{tb:mm}
\end{table}

